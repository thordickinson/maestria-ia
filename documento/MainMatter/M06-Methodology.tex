La metodología de este estudio sigue un enfoque reproducible y basado en datos abiertos para la estimación del precio de apartamentos en Bogotá. El proceso incluye adquisición e integración de datos, análisis exploratorio, limpieza y transformación, enriquecimiento geoespacial, entrenamiento de modelos base y aumentados, selección del modelo final y la implementación de una utilidad web para su uso práctico. La Figura~\ref{fig:metodologia_general} presenta una vista general del flujo metodológico.

\begin{figure}[ht]
    \centering
    \includegraphics[width=0.95\linewidth]{Images/metodologia/overview.png}
    \caption{Flujo general de la metodología desarrollada.}
    \label{fig:metodologia_general}
\end{figure}

\subsection{Adquisición e integración de datos}

El conjunto de datos principal se obtuvo a partir de una recopilación pública de inmuebles publicada en 2024. Este dataset incluye información estructural como área, baños, parqueaderos, estrato, antigüedad, estado, administración y coordenadas geográficas. Con el fin de reducir la heterogeneidad tipológica, el análisis se restringió exclusivamente a \textbf{apartamentos}, que representaban la categoría con mejor disponibilidad y coherencia estructural.

Para complementar la información base, se integraron capas oficiales del portal de Datos Abiertos de Bogotá, incluyendo barrios, UPZ, localidades, estratos y avalúos catastrales. Asimismo, se incorporaron puntos de interés (POIs) provenientes de OpenStreetMap. Todas las capas espaciales fueron cargadas en una base de datos PostGIS, lo que permitió realizar operaciones geográficas como intersecciones, cálculos de distancias y agregaciones por geohash.

\subsection{Exploración y análisis inicial de los datos}

El análisis exploratorio incluyó visualizaciones iniciales de las variables principales. Las Figuras~\ref{fig:area_log} y \ref{fig:precio_log} presentan la distribución original y transformada del área y del precio. Se observó una marcada asimetría positiva en ambas variables, lo que justificó la aplicación de la transformación logarítmica para estabilizar la varianza y mejorar la linealidad de sus relaciones.

\begin{figure}[ht]
    \centering
    \includegraphics[width=0.85\linewidth]{Images/metodologia/area_log.png}
    \caption{Distribución del área antes y después de la transformación logarítmica.}
    \label{fig:area_log}
\end{figure}

\begin{figure}[ht]
    \centering
    \includegraphics[width=0.85\linewidth]{Images/metodologia/precio_log.png}
    \caption{Distribución del precio antes y después de la transformación logarítmica.}
    \label{fig:precio_log}
\end{figure}

Se analizó la relación entre variables estructurales mediante un mapa de calor de correlaciones (Figura~\ref{fig:correlacion}). Este análisis mostró que el área y el estrato tenían la mayor correlación con el precio, mientras que variables como habitaciones, ascensor o piscina tenían correlaciones bajas, anticipando su menor utilidad en la predicción final.

\begin{figure}[ht]
    \centering
    \includegraphics[width=0.85\linewidth]{Images/metodologia/correlacion_heatmap.png}
    \caption{Mapa de calor de correlaciones entre variables estructurales.}
    \label{fig:correlacion}
\end{figure}

\subsection{Limpieza, imputación y transformación}

El dataset fue sometido a un proceso riguroso de depuración. Se eliminaron registros con coordenadas fuera del área urbana de Bogotá y valores atípicos definidos mediante percentiles (área $> 464\ \text{m}^2$ y precio $> 5.4\times10^9$ COP). Se descartaron registros inconsistentes con precios inferiores a 50 millones de pesos.

La imputación siguió una estrategia jerárquica: el área se imputó mediante la mediana de comparables dentro del mismo estrato, número de baños, parqueaderos y sector. El estrato y estado se imputaron con la moda del sector, mientras que la administración se imputó con la mediana del estrato. Variables con más de 70\% de ausencia fueron descartadas.

El precio se transformó aplicando el logaritmo natural, definiendo como variable objetivo \texttt{precio\_venta\_log}. La variable área también fue transformada logarítmicamente debido a su asimetría. Las variables numéricas se escalaron mediante \texttt{StandardScaler} y las categóricas se codificaron mediante \texttt{OneHotEncoder}.

\subsection{Enriquecimiento geoespacial}

El enriquecimiento geoespacial permitió capturar efectos del entorno urbano. Se utilizó geohash nivel 7 para agrupar propiedades dentro de celdas de aproximadamente 150 metros. A partir de las coordenadas, se asignaron automáticamente el \textbf{barrio} y la \textbf{UPZ}, que posteriormente demostraron ser variables de alta relevancia predictiva.

La Figura~\ref{fig:upz_precio} muestra un ejemplo de la relación entre UPZ y el precio promedio de los inmuebles, evidenciando la importancia territorial de esta variable.

\begin{figure}[ht]
    \centering
    \includegraphics[width=0.85\linewidth]{Images/metodologia/upz_precio.png}
    \caption{Precio promedio por UPZ en Bogotá.}
    \label{fig:upz_precio}
\end{figure}

También se calcularon conteos de POIs por radios de 100 a 2000 metros, como ilustrado en la Figura~\ref{fig:pois_radio}, que muestra la agregación espacial de servicios cercanos en un inmueble ejemplo.

\begin{figure}[ht]
    \centering
    \includegraphics[width=0.85\linewidth]{Images/metodologia/pois_radio.png}
    \caption{Ejemplo de conteo de puntos de interés en radios de proximidad (captura de la utilidad web).}
    \label{fig:pois_radio}
\end{figure}

\subsection{Construcción del modelo base}

El modelo base se desarrolló utilizando únicamente variables estructurales. Se evaluaron siete algoritmos: regresión lineal, Ridge, Lasso, SVR, Random Forest, LightGBM y XGBoost. Las métricas se calcularon mediante validación cruzada de diez pliegues.

Los resultados se resumen en la Tabla~\ref{tab:modelos_base} y la Figura~\ref{fig:rmse_modelos}.

\begin{table}[ht]
\centering
\caption{Comparación de algoritmos en el modelo base.}
\label{tab:modelos_base}
\begin{tabular}{lccc}
\hline
\textbf{Modelo} & \textbf{RMSE} & \textbf{MAE} & \textbf{$R^2$} \\
\hline
XGBoost & 0.1514 & 0.1073 & 0.9650 \\
LightGBM & 0.1604 & 0.1180 & 0.9607 \\
Random Forest & 0.1612 & 0.1121 & 0.9603 \\
\hline
\end{tabular}
\end{table}

\begin{figure}[ht]
    \centering
    \includegraphics[width=0.8\linewidth]{Images/metodologia/rmse_modelos_base.png}
    \caption{Desempeño comparado de algoritmos del modelo base (RMSE).}
    \label{fig:rmse_modelos}
\end{figure}

El mejor desempeño fue obtenido por XGBoost, que se seleccionó como modelo base oficial.

\subsection{Construcción del modelo aumentado}

El modelo aumentado incorporó todas las variables geoespaciales (POIs, geohash, avalúo catastral, barrio y UPZ). Aunque visualmente las variables espaciales son informativas, su aporte predictivo resultó marginal. La Figura~\ref{fig:importancia_aumentado} muestra la importancia de variables del modelo aumentado.

\begin{figure}[ht]
    \centering
    \includegraphics[width=0.8\linewidth]{Images/metodologia/importance_modelo_aumentado.png}
    \caption{Importancia de variables en el modelo aumentado.}
    \label{fig:importancia_aumentado}
\end{figure}

\subsection{Modelo reducido y selección final}

A partir del análisis de importancias y permutation importance, se seleccionaron once variables para construir un modelo reducido más interpretable. La Figura~\ref{fig:importancia_reducido} presenta la importancia relativa de estas variables.

\begin{figure}[ht]
    \centering
    \includegraphics[width=0.8\linewidth]{Images/metodologia/importance_modelo_reducido.png}
    \caption{Permutation importance del modelo reducido.}
    \label{fig:importancia_reducido}
\end{figure}

El modelo reducido tuvo un buen desempeño pero no superó al modelo base. La Figura~\ref{fig:error_precio} muestra el error relativo según el precio real.

\begin{figure}[ht]
    \centering
    \includegraphics[width=0.85\linewidth]{Images/metodologia/error_vs_precio.png}
    \caption{Error relativo del modelo según rango de precios reales.}
    \label{fig:error_precio}
\end{figure}

\subsection{Implementación del sistema web}

El sistema final integra un \textbf{backend en FastAPI} y un \textbf{frontend en ReactJS}. El usuario ingresa características básicas del apartamento y una dirección. La dirección se geocodifica mediante Nominatim para obtener la coordenada final, que se utiliza para asignar automáticamente el barrio y la UPZ mediante PostGIS. Esta información se envía al modelo reducido optimizado, y el backend retorna la estimación junto con estadísticas del entorno.

\subsection{Reproducibilidad}

El proyecto incluye notebooks organizados en \texttt{analisis/notebooks}, scripts modulares en \texttt{indexador-py} y artefactos almacenados en \texttt{analisis/data/}. Se fijaron semillas aleatorias para permitir replicar todos los resultados y se documentó el proceso completo de enriquecimiento y modelado.

