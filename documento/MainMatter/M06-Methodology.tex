La metodología de este estudio sigue un enfoque reproducible y basado en datos abiertos para la estimación del precio de apartamentos en Bogotá. El proceso incluye adquisición e integración de datos, análisis exploratorio, limpieza y transformación, enriquecimiento geoespacial, entrenamiento de modelos base y aumentados, selección del modelo final y la implementación de una utilidad web para su uso práctico. La Figura~\ref{fig:metodologia_general} presenta una vista general del flujo metodológico.

\begin{figure}[ht]
    \centering
    \includegraphics[width=0.95\linewidth]{Images/metodologia/overview.png}
    \caption{Flujo general de la metodología desarrollada.}
    \label{fig:metodologia_general}
\end{figure}

\section{Adquisición e integración de datos}

El conjunto de datos principal se obtuvo a partir de una recopilación pública de inmuebles publicada en 2024. Este dataset incluye información estructural como área, baños, parqueaderos, estrato, antigüedad, estado, administración y coordenadas geográficas. Con el fin de reducir la heterogeneidad tipológica, el análisis se restringió exclusivamente a \textbf{apartamentos}, que representaban la categoría con mejor disponibilidad y coherencia estructural.

Durante las etapas iniciales se consideró la posibilidad de complementar esta información mediante técnicas de \textit{web scraping} aplicadas a portales inmobiliarios nacionales. Sin embargo, estos sitios emplean mecanismos estrictos de protección contra automatización y condiciones de uso que restringen la recolección automática de datos, lo cual limita la viabilidad técnica y legal de este enfoque. Por esta razón, y con el objetivo de garantizar transparencia y reproducibilidad, el estudio se basó exclusivamente en fuentes abiertas y en conjuntos de datos disponibles públicamente.

Para complementar la información base, se integraron capas oficiales del portal de Datos Abiertos de Bogotá, incluyendo barrios, UPZ, localidades, estratos y avalúos catastrales. Asimismo, se incorporaron puntos de interés (POIs) provenientes de OpenStreetMap. Todas las capas espaciales fueron cargadas en una base de datos PostGIS, lo que permitió realizar operaciones geográficas como intersecciones, cálculos de distancias y agregaciones por geohash.

\section{Exploración y análisis inicial de los datos}

El análisis exploratorio incluyó visualizaciones iniciales de las variables principales. Las Figuras~\ref{fig:area_log} y \ref{fig:precio_log} presentan la distribución original y transformada del área y del precio. Se observó una marcada asimetría positiva en ambas variables, lo que justificó la aplicación de la transformación logarítmica para estabilizar la varianza y mejorar la linealidad de sus relaciones.

\begin{figure}[ht]
    \centering
    \includegraphics[width=0.85\linewidth]{Images/metodologia/area_log.png}
    \caption{Distribución del área antes y después de la transformación logarítmica.}
    \label{fig:area_log}
\end{figure}

\begin{figure}[ht]
    \centering
    \includegraphics[width=0.85\linewidth]{Images/metodologia/precio_log.png}
    \caption{Distribución del precio antes y después de la transformación logarítmica.}
    \label{fig:precio_log}
\end{figure}

Se analizó la relación entre variables estructurales mediante un mapa de calor de correlaciones (Figura~\ref{fig:correlacion}). Este análisis mostró que el área y el estrato tenían la mayor correlación con el precio, mientras que variables como habitaciones, ascensor o piscina tenían correlaciones bajas, anticipando su menor utilidad en la predicción final.

\begin{figure}[ht]
    \centering
    \includegraphics[width=0.85\linewidth]{Images/metodologia/correlacion_heatmap.png}
    \caption{Mapa de calor de correlaciones entre variables estructurales.}
    \label{fig:correlacion}
\end{figure}

\section{Limpieza, imputación y transformación}

El dataset fue sometido a un proceso riguroso de depuración. Se eliminaron registros con coordenadas fuera del área urbana de Bogotá y valores atípicos definidos mediante percentiles (área $> 464\ \text{m}^2$ y precio $> 5.4\times10^9$ COP). Se descartaron registros inconsistentes con precios inferiores a 50 millones de pesos.

La imputación siguió una estrategia jerárquica: el área se imputó mediante la mediana de comparables dentro del mismo estrato, número de baños, parqueaderos y sector. El estrato y estado se imputaron con la moda del sector, mientras que la administración se imputó con la mediana del estrato. Variables con más de 70\% de ausencia fueron descartadas.

El precio se transformó aplicando el logaritmo natural, definiendo como variable objetivo \texttt{precio\_venta\_log}. La variable área también fue transformada logarítmicamente debido a su asimetría. Las variables numéricas se escalaron mediante \texttt{StandardScaler} y las categóricas se codificaron mediante \texttt{OneHotEncoder}.

\section{Enriquecimiento geoespacial}

El enriquecimiento geoespacial tuvo como propósito incorporar información del entorno urbano de cada inmueble y complementar las variables estructurales con características espaciales derivadas de su localización. Para ello se utilizó geohash de nivel 7, que agrupa propiedades dentro de celdas de aproximadamente 150 metros y permite realizar agregaciones espaciales consistentes dentro del pipeline de modelado.

A partir de las coordenadas geográficas se asignaron automáticamente el \textbf{barrio}, la \textbf{UPZ} y la \textbf{localidad}, utilizando las capas oficiales de datos abiertos del Distrito. Estas variables administrativas permiten representar jerarquías territoriales relevantes para la estructura urbana, aun cuando su contribución predictiva no se asume a priori. Su incorporación se orientó principalmente a evaluar si los límites administrativos capturan diferencias espaciales que no están explícitamente presentes en las variables estructurales.

Además, se calcularon conteos de puntos de interés (POIs) dentro de radios de 100, 500, 1000 y 2000 metros, utilizando categorías de OpenStreetMap como salud, educación, comercio, recreación y servicios. La Tabla~\ref{tab:pois_radio} ilustra el proceso de agregación espacial a partir de un inmueble ejemplo, mostrando cómo se cuantifican los servicios disponibles en su entorno inmediato.

Estas variables derivadas no solo permitieron explorar la influencia del entorno urbano en la valoración inmobiliaria, sino también evaluar redundancias espaciales y determinar qué información del contexto se encuentra ya implícitamente codificada en las características estructurales y administrativas del inmueble. Este enriquecimiento constituyó la base para la posterior selección de variables mediante SHAP y la construcción del modelo optimizado.

\begin{table}[ht]
\centering
\caption{Conteo de sitios cercanos por categoría y radio}
\label{tab:pois_radio}
\begin{tabular}{lrrrrr}
\hline
\textbf{Tipo de lugar} & \textbf{100m} & \textbf{300m} & \textbf{500m} & \textbf{1000m} & \textbf{2000m} \\
\hline
Educación                & 0 & 4   & 19  & 65  & 103 \\
Alojamiento              & 0 & 28  & 57  & 99  & 119 \\
Acceso a comercio        & 0 & 31  & 74  & 221 & 492 \\
Servicios culturales     & 0 & 5   & 14  & 47  & 76  \\
Parques y recreación     & 0 & 2   & 8   & 21  & 38  \\
Servicios de infraestructura & 0 & 10 & 28 & 72 & 156 \\
Restaurantes y entretenimiento & 0 & 125 & 237 & 455 & 705 \\
Salud                    & 0 & 0   & 4   & 16  & 64  \\
Estaciones SITP          & 0 & 0   & 22  & 61  & 204 \\
Estaciones TransMilenio  & 0 & 0   & 3   & 5   & 15  \\
\hline
\end{tabular}
\end{table}


\section{Construcción del modelo base}

El modelo base se desarrolló utilizando únicamente variables estructurales. Se evaluaron siete algoritmos: regresión lineal, Ridge, Lasso, SVR, Random Forest, LightGBM y XGBoost. Las métricas se calcularon mediante validación cruzada de diez pliegues. Los resultados se resumen en la Tabla~\ref{tab:modelos_base_cop} y la Figura~\ref{fig:rmse_modelos}.

\begin{table}[ht]
\centering
\small
\caption{Comparación inicial de algoritmos de regresión (RMSE en pesos colombianos).}
\label{tab:modelos_base_cop}
\begin{tabular}{lrr}
\hline
\textbf{Modelo} & \textbf{RMSE medio (COP)} & \textbf{RMSE std (COP)} \\
\hline
Random Forest      & \$244{,}712{,}600 & \$7{,}637{,}822 \\
XGBoost            & \$245{,}248{,}800 & \$6{,}392{,}930 \\
LightGBM           & \$246{,}264{,}200 & \$4{,}892{,}081 \\
KNN                & \$341{,}664{,}900 & \$5{,}987{,}948 \\
Decision Tree      & \$345{,}986{,}400 & \$8{,}217{,}641 \\
Linear Regression  & \$347{,}985{,}800 & \$4{,}397{,}948 \\
Lasso              & \$347{,}988{,}400 & \$4{,}402{,}652 \\
Ridge              & \$347{,}989{,}900 & \$4{,}437{,}838 \\
SVR                & \$913{,}333{,}600 & \$5{,}537{,}129 \\
\hline
\end{tabular}
\end{table}

\begin{figure}[ht]
    \centering
    \includegraphics[width=0.8\linewidth]{Images/metodologia/rmse_modelos_base.png}
    \caption{Desempeño comparado de algoritmos del modelo base (RMSE).}
    \label{fig:rmse_modelos}
\end{figure}

El mejor desempeño fue obtenido por XGBoost, que se seleccionó como modelo base oficial.

\section{Construcción, evaluación e interpretabilidad del modelo enriquecido}

El modelo enriquecido tuvo como objetivo evaluar el aporte de las variables geoespaciales sobre la capacidad predictiva y explicativa del sistema. Para construir este modelo se integraron todas las variables derivadas del proceso de enriquecimiento espacial: conteos de puntos de interés (POIs) en distintos radios, identificadores administrativos (barrio y UPZ), geohash nivel~7 y el avalúo catastral zonal. Todas estas variables fueron procesadas mediante el mismo \texttt{ColumnTransformer} utilizado en el modelo base, garantizando imputación, escalamiento y codificación categórica consistentes en todas las etapas del pipeline.

Con el fin de evaluar el comportamiento del conjunto enriquecido, se entrenaron y validaron varios algoritmos utilizados previamente en el modelo base: Linear Regression, Ridge, Lasso, Random Forest, LightGBM y XGBoost. El modelo SVR se descartó desde etapas anteriores por su bajo rendimiento y alto costo computacional. La evaluación se realizó mediante validación cruzada de diez pliegues, utilizando RMSE como métrica principal. Los resultados se presentan en la Tabla~\ref{tab:resultados_aumentado}.

\begin{table}[ht]
\centering
\small
\caption{Desempeño de algoritmos en el modelo enriquecido (10-Fold CV).}
\label{tab:resultados_aumentado}
\begin{tabular}{lcccc}
\hline
\textbf{Modelo} & \textbf{RMSE} & \textbf{RMSE std} & \textbf{MAE} & \textbf{$R^2$} \\
\hline
LightGBM          & 0.171 & 0.005 & 0.126 & 0.956 \\
XGBoost           & 0.171 & 0.005 & 0.126 & 0.956 \\
Random Forest     & 0.178 & 0.004 & 0.128 & 0.952 \\
Ridge             & 0.279 & 0.008 & 0.212 & 0.881 \\
Linear Regression & 0.280 & 0.009 & 0.212 & 0.880 \\
Lasso             & 0.337 & 0.007 & 0.263 & 0.827 \\
\hline
\end{tabular}
\end{table}

La comparación inicial evidenció que LightGBM y XGBoost presentaron desempeños equivalentes. Sin embargo, se seleccionó XGBoost para las siguientes etapas del análisis por dos razones metodológicas: (i) el modelo base ya había sido entrenado y optimizado con XGBoost, lo que permite comparar ambos modelos modificando exclusivamente el conjunto de variables y no el algoritmo; y (ii) XGBoost tiene una integración más estable y documentada con SHAP, lo cual es fundamental para el análisis interpretativo que guía la selección del subconjunto final de características. Esta elección garantiza consistencia, comparabilidad y estabilidad numérica en todo el proceso de modelado.

Con el modelo enriquecido entrenado, se aplicó SHAP (\textit{SHapley Additive exPlanations}) para analizar la contribución local y global de cada característica después del preprocesamiento. La Figura~\ref{fig:shap_summary} presenta el \textit{SHAP Summary Plot}, que muestra la distribución del impacto de cada variable a nivel individual, mientras que la Figura~\ref{fig:shap_bar} resume la importancia global de las características mediante el valor SHAP absoluto promedio.

\begin{figure}[ht]
    \centering
    \includegraphics[width=0.5\linewidth]{Images/metodologia/shap_enriched_summary.png}
    \caption{SHAP Summary Plot del modelo enriquecido.}
    \label{fig:shap_summary}
\end{figure}

\begin{figure}[ht]
    \centering
    \includegraphics[width=0.5\linewidth]{Images/metodologia/shap_enriched_barplot.png}
    \caption{Importancia global de las variables según valores SHAP.}
    \label{fig:shap_bar}
\end{figure}

El análisis permitió observar qué variables del entorno urbano ofrecían contribución real y cuáles resultaban redundantes frente a las ya presentes en el modelo base. En general, las variables derivadas de POIs y de accesibilidad urbana mostraron una contribución marginal, lo cual coincide con su bajo aporte predictivo cuantitativo. En cambio, variables estructurales como área, baños, parqueaderos y administración, junto con coordenadas geográficas (latitud y longitud), concentraron la mayor parte de la señal relevante. A partir del ranking SHAP, se seleccionó un subconjunto de variables con mayor aporte promedio, lo que dio lugar al \textit{modelo enriquecido optimizado}. Este modelo se entrenó utilizando el mismo pipeline y los mismos hiperparámetros, manteniendo consistencia metodológica y permitiendo comparar directamente los efectos de la selección de variables.

Los resultados detallados se reportan en la sección de validación cruzada, donde se contrastan los tres modelos (base, enriquecido y enriquecido optimizado) mediante métricas descriptivas y pruebas estadísticas.

\section{Modelo enriquecido optimizado}

A partir del análisis de importancia mediante SHAP se seleccionó un subconjunto reducido de variables con mayor aporte promedio, compuesto principalmente por características estructurales (área, baños, parqueaderos, administración), algunos atributos administrativos (antigüedad) y componentes geográficos esenciales (latitud y longitud). La mayoría de variables geoespaciales derivadas —como conteos de POIs y categorías OSM— mostraron contribución marginal, por lo que fueron descartadas en esta versión optimizada.

El modelo se entrenó utilizando el mismo pipeline y los mismos hiperparámetros del modelo enriquecido, de modo que cualquier cambio en el desempeño se debiera exclusivamente a la selección de variables. La validación cruzada de diez pliegues produjo un RMSE promedio de 0.1584 (std 0.0082), un MAE de 0.1138 y un $R^2$ de 0.9600. En la escala monetaria original, esto corresponde a un RMSE promedio de aproximadamente \textbf{\$233 millones COP} y un MAE de \textbf{\$123 millones COP}. Estos valores ubican al modelo optimizado entre el modelo base y el modelo enriquecido, reduciendo el ruido introducido por variables geoespaciales redundantes, aunque sin alcanzar la precisión lograda por el modelo base.

Para evaluar si las diferencias eran significativas, se aplicaron pruebas t pareadas utilizando los valores de RMSE de los diez pliegues. Se encontraron diferencias significativas entre el modelo base y el enriquecido ($p = 5.49\times10^{-6}$), entre el modelo base y el optimizado ($p = 0.0112$), y entre el modelo enriquecido y el optimizado ($p = 0.000003$), indicando que reducir variables mejora la estabilidad del modelo enriquecido sin recuperar completamente la precisión del modelo base.

La Tabla~\ref{tab:comparacion_modelos_final} resume el desempeño de los tres modelos evaluados. Aunque el modelo base conserva la mayor precisión, el modelo optimizado ofrece un balance adecuado entre simplicidad, interpretabilidad y error, constituyéndose en una alternativa eficiente dentro del conjunto enriquecido.

\begin{table}[ht]
\centering
\small
\caption{Comparación de desempeño entre los modelos evaluados (10-Fold CV, valores en millones de pesos).}
\label{tab:comparacion_modelos_final}
\begin{tabular}{lccccc}
\hline
\textbf{Modelo} & \textbf{RMSE (M\$)} & \textbf{RMSE std} & 
\textbf{MAE (M\$)} & \textbf{MAE std} & \textbf{$R^2$} \\
\hline
Base                    & 231 & 8   & 120 & 3  & 0.965 \\
Enriquecido             & 257 & 12  & 138 & 5  & 0.956 \\
Enriquecido optimizado  & 233 & 14  & 123 & 5  & 0.960 \\
\hline
\end{tabular}
\end{table}


\section{Implementación del sistema web}

El sistema desarrollado integra un \textbf{backend en FastAPI}, una base de datos \textbf{PostgreSQL con PostGIS} y una interfaz \textbf{frontend en ReactJS}. Su propósito es permitir que un usuario ingrese la información básica de un inmueble y obtenga, en tiempo real, una estimación de precio junto con estadísticas del entorno urbano. La Figura~\ref{fig:arquitectura_sistema} presenta una vista general de la arquitectura implementada.

\begin{figure}[ht]
    \centering
    \includegraphics[width=0.9\linewidth]{Images/metodologia/arquitectura_app.png}
    \caption{Arquitectura general del sistema web desarrollado.}
    \label{fig:arquitectura_sistema}
\end{figure}


Durante la fase de preparación de datos, el conjunto de propiedades utilizado para el entrenamiento del modelo fue enriquecido con información geográfica y posteriormente se generaron \textbf{estadísticos agregados por barrio, UPZ y localidad}, incluyendo valores promedio de área, administración, estrato y precio estimado. Estos datos derivados se almacenaron en la base de datos PostgreSQL para permitir su recuperación eficiente cuando el usuario consulta la aplicación. Adicionalmente, el modelo final (enriquecido optimizado) fue exportado a un archivo \texttt{.pkl}, lo que facilita su carga y ejecución dentro del backend sin necesidad de reconstruir el pipeline completo en cada solicitud.

El backend recibe las solicitudes provenientes del frontend y ejecuta tres tareas principales: (i) geocodificación de la dirección ingresada mediante el servicio Nominatim; (ii) asignación automática de barrio y UPZ utilizando consultas espaciales en PostGIS; y (iii) ejecución del modelo optimizado para generar la estimación de precio. La respuesta enviada al frontend incluye tanto la predicción como un conjunto de métricas contextuales del sector, obtenidas directamente desde las tablas agregadas precomputadas en la base de datos.

En la aplicación frontend, la interacción inicia con un formulario donde el usuario ingresa las características físicas del inmueble (área, baños, parqueaderos, antigüedad, estrato y administración). En una segunda pantalla se solicita la \textbf{dirección}, que es geocodificada automáticamente para obtener las coordenadas iniciales. Estas coordenadas se muestran en un mapa interactivo implementado con \texttt{Leaflet}, permitiendo al usuario mover el marcador manualmente para ajustar la ubicación con mayor precisión si lo desea.

Una vez confirmados los datos, la aplicación presenta un \textbf{reporte interactivo}. Este reporte incluye:  
(1) un mapa con los puntos de interés cercanos al inmueble en distintos radios de distancia;  
(2) una tabla con los conteos de POIs por categoría y radio;  
(3) comparaciones visuales entre las características del inmueble y los promedios del barrio, la UPZ y la localidad;  
(4) gráficos que resumen estadísticas relevantes del entorno urbano.  
Ejemplos de estas visualizaciones se muestran en las Figuras~\ref{fig:reporte_mapa_pois} y~\ref{fig:comparacion_estadisticas}.

\begin{figure}[ht]
    \centering
    \includegraphics[width=0.85\linewidth]{Images/metodologia/reporte_mapa.png}
    \caption{Vista del reporte generado: mapa con puntos de interés cercanos.}
    \label{fig:reporte_mapa_pois}
\end{figure}

\begin{figure}[ht]
    \centering
    \includegraphics[width=0.85\linewidth]{Images/metodologia/reporte_comparativa_barrio.png}
    \caption{Comparativa entre las características del inmueble y los promedios del barrio y la UPZ.}
    \label{fig:comparacion_estadisticas}
\end{figure}

Este diseño permite que el modelo predictivo se integre en una plataforma accesible y comprensible para el usuario final, ofreciendo no solo una estimación numérica sino una contextualización completa del entorno urbano, lo cual facilita la interpretación del resultado y mejora la utilidad práctica del sistema en escenarios de consulta inmobiliaria.
