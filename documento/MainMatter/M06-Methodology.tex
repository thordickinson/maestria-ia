\section*{Metodología}

El desarrollo del proyecto se estructuró como un proceso reproducible de estimación de precios de vivienda para la ciudad de Bogotá, combinando técnicas de análisis de datos, modelado predictivo y enriquecimiento geoespacial. La metodología se organizó en etapas secuenciales, desde la adquisición de datos hasta la validación comparativa de modelos.

\begin{enumerate}
    \item Adquisición y validación de datos.
    \item Limpieza, imputación y normalización de variables.
    \item Entrenamiento del modelo base y evaluación de algoritmos.
    \item Enriquecimiento geoespacial mediante capas distritales y puntos de interés (POIs).
    \item Optimización de hiperparámetros y análisis de importancia de variables.
    \item Evaluación cruzada, análisis de error y comparación final de modelos.
    \item Exposición de resultados y persistencia mediante API.
\end{enumerate}

La métrica principal seleccionada fue la \textbf{raíz del error cuadrático medio (RMSE)}, por su interpretabilidad en la misma unidad del objetivo (pesos colombianos) y su sensibilidad a errores grandes, deseable en contextos de valoración. Como métricas complementarias se reportan el \textbf{error absoluto medio (MAE)} y el \(\mathbf{R^2}\), indicador del grado de ajuste explicativo.

\subsection*{Fuentes de datos}

Se integraron fuentes públicas y abiertas que combinan información estructural del inmueble con variables contextuales espaciales. 

\begin{itemize}
    \item \textbf{Datos de inmuebles (base):} Se utilizaron registros provenientes del repositorio público \texttt{builker-col/bogota-apartments} \cite{portales_inmobiliarios}, con datos correspondientes a agosto de 2024. Los archivos originales en formato JSON fueron transformados a CSV para su procesamiento.
    \item \textbf{Datos abiertos del Distrito (PostGIS):} Se descargaron capas geoespaciales desde el portal de datos abiertos de la Alcaldía de Bogotá \cite{datosabiertos_bogota}, incluyendo \texttt{barrios\_bogota}, \texttt{upz\_bogota}, \texttt{localidades\_bogota}, \texttt{estratos\_manzana}, \texttt{avaluo\_catastral\_manzana} y \texttt{avaluo\_comercial\_manzana}. Estas capas fueron cargadas en una base de datos PostGIS mediante el uso de la herramienta shp2pgsql.
    \item \textbf{Puntos de interés (POIs):} Se obtuvieron datos de OpenStreetMap descargados desde Geofabrik \cite{geofabrik}. Se procesaron las categorías \emph{education}, \emph{healthcare}, \emph{retail\_access}, \emph{dining}, \emph{accommodation}, \emph{parks}, \emph{infrastructure} y \emph{culture}, calculando su cercanía a cada inmueble en radios de 100 a 2000 metros.
\end{itemize}

\subsection*{Limpieza y preprocesamiento}

Se ejecutaron rutinas de limpieza orientadas a garantizar la coherencia estructural de las variables y reducir la influencia de errores de captura o valores extremos. El procesamiento se realizó en Python utilizando \textit{pandas}, \textit{NumPy} y \textit{scikit-learn}.

\begin{itemize}
    \item \textbf{Outliers:} Se eliminaron observaciones por encima del percentil 99 en área y precio. \(\textit{Área} \leq 464\,m^2\), \(\textit{precio\_venta} \leq 5{,}4\times10^9\) COP.
    \item \textbf{Valores mínimos:} Se excluyeron registros con precios inferiores a 50 millones de pesos.
    \item \textbf{Imputación:} Se imputaron valores faltantes de \textit{área}, \textit{estrato}, \textit{estado} y \textit{administración} por mediana o moda dentro del mismo sector o estrato.
    \item \textbf{Coordenadas:} Se validó que las coordenadas estuvieran dentro del bounding box de Bogotá (latitud \([4.4, 4.9]\); longitud \([-74.3, -73.9]\)).
\end{itemize}

\subsection*{Transformaciones y normalización de variables}

Dada la asimetría positiva de las distribuciones de \textit{área} y \textit{precio\_venta}, se aplicó una transformación logarítmica para estabilizar la varianza y mejorar la linealidad en las relaciones con el objetivo.  
La Figura~\ref{fig:log_area_precio} muestra la diferencia en la distribución de ambas variables antes y después de la transformación.

\begin{figure}[h]
    \centering
    \includegraphics[width=0.8\linewidth]{Images/log_area_precio.png}
    \caption{Distribución del área y precio antes y después de la transformación logarítmica}
    \label{fig:log_area_precio}
\end{figure}


\subsection*{Uso de geohash y enriquecimiento geoespacial}

Para agregar información contextual de forma eficiente, se utilizó la codificación \textbf{geohash} con nivel de precisión 7, correspondiente a celdas de aproximadamente 150 metros \cite{geohash_size}.  
Esto permitió agrupar propiedades próximas y calcular estadísticas agregadas de variables geográficas como estrato, valor catastral y densidad de puntos de interés.  
El enriquecimiento geoespacial se implementó sobre PostGIS mediante funciones \texttt{ST\_Contains} y \texttt{ST\_Distance}, generando nuevas variables para cada inmueble.

\begin{itemize}
    \item Conteo de POIs por categorías OSM en radios de 100, 300, 500, 1000 y 2000 metros.
    \item Asignación de variables \textit{barrio\_calculado}, \textit{upz\_calculada} y \textit{localidad\_calculada}.
    \item Estadísticos agregados (\(n\), media, desviación y cuartiles) por celda geohash y por región administrativa.
\end{itemize}

\subsection*{Evaluación de modelos y validación cruzada}

Se evaluaron diferentes algoritmos de regresión: regresión lineal, Ridge, Lasso, SVR, Random Forest, XGBoost y LightGBM.  
Todos los modelos compartieron un pipeline común de preprocesamiento (\texttt{SimpleImputer}, \texttt{StandardScaler}, \texttt{OneHotEncoder}) y fueron validados mediante \textbf{validación cruzada de 10 pliegues}.

El modelo \textbf{XGBoost} presentó el mejor desempeño en el conjunto base, con \(\text{RMSE} = 0.151\), \(\text{MAE} = 0.107\) y \(R^2 = 0.965\).  
Este modelo se seleccionó como línea base para la comparación con los modelos enriquecidos.

\subsection*{Optimización e interpretabilidad}

Se aplicó búsqueda aleatoria de hiperparámetros (\texttt{RandomizedSearchCV}, 5x25) para optimizar los parámetros del modelo base.  
Los mejores valores obtenidos fueron \(n\_\textit{estimators}=627\), \(\textit{max\_depth}=9\), \(\textit{learning\_rate}=0.05\), \(\textit{subsample}=0.85\) y \(\textit{colsample\_bytree}=0.90\).  
Tras la optimización, el modelo mantuvo un RMSE similar (0.158), lo que confirmó su estabilidad.

El análisis de importancia de variables mostró que las características estructurales dominan la predicción: \textbf{área} (0.29), \textbf{estrato} (0.15), \textbf{parqueaderos} (0.11) y \textbf{antigüedad} (0.06) son los factores más determinantes.  
Las variables de amenidades y ubicación (latitud, longitud, piscina, gimnasio) aportan información adicional, pero con influencia menor.

\subsection*{Análisis de error}

El análisis de error por rangos de precio mostró que el error absoluto aumenta con el valor del inmueble, pero el error relativo se mantiene estable (3--4\%), lo que evidencia consistencia en todo el rango de precios.  
Por sectores, los errores más altos se concentraron en zonas de alta valorización (Chicó, Santa Cecilia, Cerros de Suba), debido a su heterogeneidad.  
Este patrón justifica la incorporación de variables geoespaciales para capturar diferencias locales no explicadas por las variables estructurales.

\subsection*{Comparación final de modelos}

El modelo \textbf{enriquecido reducido} (con variables geoespaciales) obtuvo un \(\text{RMSE} = 0.165\), \(\text{MAE} = 0.121\) y \(R^2 = 0.958\), mientras que el modelo base mantuvo \(\text{RMSE} = 0.151\) y \(R^2 = 0.965\).  
La diferencia relativa del 9.2\% en RMSE y del -0.7\% en \(R^2\) indica que el enriquecimiento geoespacial no mejora sustancialmente el rendimiento, pero aporta valor explicativo adicional y mejora la interpretabilidad espacial del modelo.

\subsection*{Implementación del sistema web}

Además del modelo analítico, se desarrolló una \textbf{arquitectura web funcional} compuesta por un servicio \textit{backend} en Python (FastAPI) y una interfaz \textit{frontend} en ReactJS. El \textit{backend} expone un conjunto de endpoints REST que permiten no solo estimar el precio de un apartamento a partir de sus características estructurales, sino también consultar \textbf{estadísticas contextuales} sobre el sector correspondiente, incluyendo precios promedio, estratos predominantes, proximidad a servicios y conteo de amenidades registradas en OpenStreetMap.

El \textit{frontend}, desarrollado con ReactJS, ofrece una interfaz interactiva que facilita la introducción de las variables del inmueble (área, número de baños, parqueaderos, estrato, barrio, entre otros). La aplicación envía estos datos al servicio FastAPI, el cual los enriquece con información geoespacial y devuelve la predicción generada por el modelo final optimizado y con características reducidas. Esta integración asegura la \textbf{usabilidad práctica del sistema} y su potencial aplicación como herramienta de consulta para agentes, ciudadanos y analistas del mercado inmobiliario.


