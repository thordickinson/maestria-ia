La metodología de este estudio sigue un enfoque reproducible y basado en datos abiertos para la estimación del precio de apartamentos en Bogotá. El proceso incluye adquisición e integración de datos, análisis exploratorio, limpieza y transformación, enriquecimiento geoespacial, entrenamiento de modelos base y aumentados, selección del modelo final y la implementación de una utilidad web para su uso práctico. La Figura~\ref{fig:metodologia_general} presenta una vista general del flujo metodológico.

\begin{figure}[ht]
    \centering
    \includegraphics[width=0.95\linewidth]{Images/metodologia/overview.png}
    \caption{Flujo general de la metodología desarrollada.}
    \label{fig:metodologia_general}
\end{figure}

\subsection{Adquisición e integración de datos}

El conjunto de datos principal se obtuvo a partir de una recopilación pública de inmuebles publicada en 2024. Este dataset incluye información estructural como área, baños, parqueaderos, estrato, antigüedad, estado, administración y coordenadas geográficas. Con el fin de reducir la heterogeneidad tipológica, el análisis se restringió exclusivamente a \textbf{apartamentos}, que representaban la categoría con mejor disponibilidad y coherencia estructural.

Para complementar la información base, se integraron capas oficiales del portal de Datos Abiertos de Bogotá, incluyendo barrios, UPZ, localidades, estratos y avalúos catastrales. Asimismo, se incorporaron puntos de interés (POIs) provenientes de OpenStreetMap. Todas las capas espaciales fueron cargadas en una base de datos PostGIS, lo que permitió realizar operaciones geográficas como intersecciones, cálculos de distancias y agregaciones por geohash.

\subsection{Exploración y análisis inicial de los datos}

El análisis exploratorio incluyó visualizaciones iniciales de las variables principales. Las Figuras~\ref{fig:area_log} y \ref{fig:precio_log} presentan la distribución original y transformada del área y del precio. Se observó una marcada asimetría positiva en ambas variables, lo que justificó la aplicación de la transformación logarítmica para estabilizar la varianza y mejorar la linealidad de sus relaciones.

\begin{figure}[ht]
    \centering
    \includegraphics[width=0.85\linewidth]{Images/metodologia/area_log.png}
    \caption{Distribución del área antes y después de la transformación logarítmica.}
    \label{fig:area_log}
\end{figure}

\begin{figure}[ht]
    \centering
    \includegraphics[width=0.85\linewidth]{Images/metodologia/precio_log.png}
    \caption{Distribución del precio antes y después de la transformación logarítmica.}
    \label{fig:precio_log}
\end{figure}

Se analizó la relación entre variables estructurales mediante un mapa de calor de correlaciones (Figura~\ref{fig:correlacion}). Este análisis mostró que el área y el estrato tenían la mayor correlación con el precio, mientras que variables como habitaciones, ascensor o piscina tenían correlaciones bajas, anticipando su menor utilidad en la predicción final.

\begin{figure}[ht]
    \centering
    \includegraphics[width=0.85\linewidth]{Images/metodologia/correlacion_heatmap.png}
    \caption{Mapa de calor de correlaciones entre variables estructurales.}
    \label{fig:correlacion}
\end{figure}

\subsection{Limpieza, imputación y transformación}

El dataset fue sometido a un proceso riguroso de depuración. Se eliminaron registros con coordenadas fuera del área urbana de Bogotá y valores atípicos definidos mediante percentiles (área $> 464\ \text{m}^2$ y precio $> 5.4\times10^9$ COP). Se descartaron registros inconsistentes con precios inferiores a 50 millones de pesos.

La imputación siguió una estrategia jerárquica: el área se imputó mediante la mediana de comparables dentro del mismo estrato, número de baños, parqueaderos y sector. El estrato y estado se imputaron con la moda del sector, mientras que la administración se imputó con la mediana del estrato. Variables con más de 70\% de ausencia fueron descartadas.

El precio se transformó aplicando el logaritmo natural, definiendo como variable objetivo \texttt{precio\_venta\_log}. La variable área también fue transformada logarítmicamente debido a su asimetría. Las variables numéricas se escalaron mediante \texttt{StandardScaler} y las categóricas se codificaron mediante \texttt{OneHotEncoder}.

\subsection{Enriquecimiento geoespacial}

El enriquecimiento geoespacial permitió capturar efectos del entorno urbano. Se utilizó geohash nivel 7 para agrupar propiedades dentro de celdas de aproximadamente 150 metros. A partir de las coordenadas, se asignaron automáticamente el \textbf{barrio} y la \textbf{UPZ}, que posteriormente demostraron ser variables de alta relevancia predictiva.

La Figura~\ref{fig:upz_precio} muestra un ejemplo de la relación entre UPZ y el precio promedio de los inmuebles, evidenciando la importancia territorial de esta variable.

\begin{figure}[ht]
    \centering
    \includegraphics[width=0.85\linewidth]{Images/metodologia/upz_precio.png}
    \caption{Precio promedio por UPZ en Bogotá.}
    \label{fig:upz_precio}
\end{figure}

También se calcularon conteos de POIs por radios de 100 a 2000 metros, como ilustrado en la Figura~\ref{fig:pois_radio}, que muestra la agregación espacial de servicios cercanos en un inmueble ejemplo.

\begin{figure}[ht]
    \centering
    \includegraphics[width=0.85\linewidth]{Images/metodologia/pois_radio.png}
    \caption{Ejemplo de conteo de puntos de interés en radios de proximidad (captura de la utilidad web).}
    \label{fig:pois_radio}
\end{figure}

\subsection{Construcción del modelo base}

El modelo base se desarrolló utilizando únicamente variables estructurales. Se evaluaron siete algoritmos: regresión lineal, Ridge, Lasso, SVR, Random Forest, LightGBM y XGBoost. Las métricas se calcularon mediante validación cruzada de diez pliegues.

Los resultados se resumen en la Tabla~\ref{tab:modelos_base} y la Figura~\ref{fig:rmse_modelos}.

\begin{table}[ht]
\centering
\caption{Comparación de algoritmos en el modelo base.}
\label{tab:modelos_base}
\begin{tabular}{lccc}
\hline
\textbf{Modelo} & \textbf{RMSE} & \textbf{MAE} & \textbf{$R^2$} \\
\hline
XGBoost & 0.1514 & 0.1073 & 0.9650 \\
LightGBM & 0.1604 & 0.1180 & 0.9607 \\
Random Forest & 0.1612 & 0.1121 & 0.9603 \\
\hline
\end{tabular}
\end{table}

\begin{figure}[ht]
    \centering
    \includegraphics[width=0.8\linewidth]{Images/metodologia/rmse_modelos_base.png}
    \caption{Desempeño comparado de algoritmos del modelo base (RMSE).}
    \label{fig:rmse_modelos}
\end{figure}

El mejor desempeño fue obtenido por XGBoost, que se seleccionó como modelo base oficial.

\subsection{Construcción del modelo enriquecido}

El modelo enriquecido incorporó todas las variables geoespaciales (POIs, geohash, avalúo catastral, barrio y UPZ). Aunque visualmente las variables espaciales son informativas, su aporte predictivo resultó marginal. La Figura~\ref{fig:importancia_aumentado} muestra la importancia de variables del modelo aumentado.

\begin{figure}[ht]
    \centering
    \includegraphics[width=0.8\linewidth]{Images/metodologia/importance_modelo_aumentado.png}
    \caption{Importancia de variables en el modelo aumentado.}
    \label{fig:importancia_aumentado}
\end{figure}

\subsection{Interpretabilidad del Modelo mediante SHAP}

Para analizar la contribuci\'on relativa de todas las variables del modelo enriquecido,
se emplearon valores SHAP. Tras aplicar el preprocesamiento, se generaron:

\begin{itemize}
    \item \textbf{Figura A}: \textit{SHAP Summary Plot},
    \item \textbf{Figura B}: \textit{SHAP Bar Plot}.
\end{itemize}

Estas figuras permiten identificar qu\'e variables tienen mayor impacto en la predicci\'on 
y guiar la selecci\'on de caracter\'isticas para un modelo m\'as parsimonioso.


\subsection{Construcci\'on del Modelo Enriquecido Optimizado}

A partir del ranking global de importancia obtenido con SHAP, se seleccion\'o un subconjunto
de caracter\'isticas con mayor aporte promedio. Las variables finalmente retenidas se listan en la Tabla~\ref{tab:variables_optimizadas}.

Con estas variables se entren\'o el \textit{modelo enriquecido optimizado},
utilizando el mismo esquema de preprocesamiento y los mismos hiperpar\'ametros del modelo enriquecido.


\subsection{Validaci\'on cruzada y m\'etricas de evaluaci\'on}

La evaluaci\'on de los modelos se realiz\'o mediante \textit{validaci\'on cruzada de diez pliegues} (10-Fold Cross-Validation). 
En cada iteraci\'on, nueve pliegues se emplean para el entrenamiento y el pliegue restante para validaci\'on, de forma que 
todas las observaciones act\'uan como conjunto de prueba exactamente una vez.

Para cada pliegue se registraron las m\'etricas:
\begin{itemize}
    \item \textbf{RMSE}: Ra\'iz del error cuadr\'atico medio.
    \item \textbf{MAE}: Error absoluto medio.
    \item \textbf{R\textsuperscript{2}}: Coeficiente de determinaci\'on.
\end{itemize}

Los resultados detallados de cada modelo, incluyendo los diez valores individuales por pliegue y las estad\'isticas descriptivas 
(media, desviaci\'on est\'andar, percentiles, m\'inimo y m\'aximo), se presentan en:

\begin{itemize}
    \item \textbf{Tabla X}: Resultados del modelo base.
    \item \textbf{Tabla Y}: Resultados del modelo enriquecido.
    \item \textbf{Tabla Z}: Resultados del modelo enriquecido optimizado.
\end{itemize}

Este esquema permite comparar estabilidad, varianza y consistencia del desempe\~no entre modelos.

\subsection{Comparaci\'on estad\'istica entre modelos}

Para determinar si las diferencias de desempe\~no entre los modelos son estad\'isticamente significativas, se aplicaron 
\textit{pruebas t de Student para muestras pareadas}, utilizando los diez valores de RMSE obtenidos en cada pliegue para cada modelo.

Se evaluaron los siguientes contrastes:
\begin{enumerate}
    \item Modelo base vs.\ modelo enriquecido.
    \item Modelo base vs.\ modelo enriquecido optimizado.
    \item Modelo enriquecido vs.\ modelo enriquecido optimizado.
\end{enumerate}

Los resultados del estad\'istico t y los p-valores asociados se presentan en la \textbf{Tabla W}, junto con una interpretaci\'on detallada 
de los contrastes realizados.

\subsection{Interpretabilidad del modelo mediante SHAP}

El modelo enriquecido fue analizado utilizando valores SHAP (\textit{SHapley Additive exPlanations}), con el fin de identificar 
las variables con mayor contribuci\'on a la explicaci\'on del modelo.

Se generaron dos visualizaciones principales:
\begin{itemize}
    \item \textbf{Figura A}: \textit{SHAP summary plot}, que muestra la distribuci\'on del impacto de cada variable.
    \item \textbf{Figura B}: \textit{SHAP bar plot}, que presenta la importancia media absoluta de cada caracter\'istica.
\end{itemize}

Estas figuras permiten analizar el aporte relativo de las variables estructurales y geoespaciales, as\'i como su influencia 
sobre la predicci\'on del precio de los apartamentos.

\subsection{Construcci\'on del modelo enriquecido optimizado}

A partir de los valores SHAP obtenidos (Figura B), se seleccionaron las variables con contribuci\'on significativa seg\'un su 
importancia global. Las caracter\'isticas retenidas se listan en la \textbf{Tabla Q}, junto con su valor de importancia media.

Con este conjunto reducido de variables se entren\'o el \textit{modelo enriquecido optimizado}, utilizando el mismo 
preprocesamiento basado en \texttt{ColumnTransformer} y los mismos hiperpar\'ametros ajustados previamente.

Los resultados completos del proceso de validaci\'on cruzada para este modelo se presentan en la \textbf{Tabla Z}, y su 
comparaci\'on estad\'istica frente a los otros modelos en la \textbf{Tabla W}.

\subsection{S\'intesis comparativa final}

Para facilitar la interpretaci\'on global del desempe\~no de los modelos, en la \textbf{Tabla R} se presentan de manera consolidada 
las m\'etricas:

\begin{itemize}
    \item RMSE medio $\pm$ desviaci\'on est\'andar.
    \item MAE medio $\pm$ desviaci\'on est\'andar.
    \item R\textsuperscript{2} medio $\pm$ desviaci\'on est\'andar.
    \item p-valores de las pruebas t correspondientes.
\end{itemize}

Esta tabla resume la evidencia cuantitativa necesaria para analizar la contribuci\'on del enriquecimiento de datos, 
la utilidad del modelo optimizado y la comparaci\'on final entre los enfoques desarrollados.

\subsection{Implementación del sistema web}

El sistema final integra un \textbf{backend en FastAPI} y un \textbf{frontend en ReactJS}. El usuario ingresa características básicas del apartamento y una dirección. La dirección se geocodifica mediante Nominatim para obtener la coordenada final, que se utiliza para asignar automáticamente el barrio y la UPZ mediante PostGIS. Esta información se envía al modelo reducido optimizado, y el backend retorna la estimación junto con estadísticas del entorno.

\subsection{Reproducibilidad}

El proyecto incluye notebooks organizados en \texttt{analisis/notebooks}, scripts modulares en \texttt{indexador-py} y artefactos almacenados en \texttt{analisis/data/}. Se fijaron semillas aleatorias para permitir replicar todos los resultados y se documentó el proceso completo de enriquecimiento y modelado.

