% En ella, el autor presenta y señala la importancia, el origen (los antecedentes teóricos y prácticos), los objetivos, los alcances, las limitaciones, la metodología empleada, el significado que el estudio tiene en el avance del campo respectivo y su aplicación en el área investigada.

% Entre 60 y 80 páginas

La vivienda es un componente central del bienestar social y del desarrollo urbano. Su localización, calidad y accesibilidad influyen en la equidad, la movilidad y la estructura socioespacial de las ciudades. En Colombia, el déficit habitacional y las desigualdades en el acceso a vivienda digna siguen siendo retos persistentes. Como señala Castillo (2004), este déficit no solo se relaciona con la cantidad de viviendas disponibles, sino también con las condiciones del entorno y la provisión de servicios urbanos \cite{castillo2004anotaciones}. En Bogotá, procesos históricos de urbanización informal y segregación han reforzado patrones de exclusión y fragmentación territorial.

La dificultad de acceso a crédito, la informalidad en el mercado y la especulación inmobiliaria contribuyen a incrementar los precios y profundizar las brechas sociales. Más del 80\,\% de los hogares de bajos ingresos no cuenta con financiamiento formal \cite{castillo2004anotaciones}, mientras que políticas como \textit{Mi Casa Ya} han tenido implementaciones irregulares que afectan la estabilidad del mercado \cite{coyunturaVivienda2023}. Paralelamente, la ausencia de herramientas analíticas transparentes limita la capacidad de comprender y monitorear la dinámica real de los precios.

Tradicionalmente, la valoración de inmuebles en Bogotá se ha basado en el juicio experto y en comparaciones simples, especialmente el \textit{precio por metro cuadrado}, ajustado por características adicionales. Aunque esta metodología refleja prácticas del sector, carece de una integración sistemática de factores espaciales, económicos y urbanos. En contraste, la literatura internacional ha mostrado que técnicas de aprendizaje automático permiten capturar relaciones no lineales y aprovechar grandes volúmenes de datos estructurales y geoespaciales. Estudios como \cite{dabreo2021realestate}, \cite{bigdata2019realestate} y \cite{kim2018machinelearning} demuestran la efectividad de modelos como XGBoost, LightGBM y redes neuronales en la predicción de precios. Sin embargo, su aplicación en contextos latinoamericanos sigue siendo limitada por la fragmentación de datos y la escasez de información espacial estandarizada.

En Colombia, los modelos automatizados se basan principalmente en información proveniente de portales inmobiliarios, los cuales presentan sesgos derivados de su modelo de negocio: la publicación tiene costo y favorece inmuebles de mayor valor, generando una representación parcial de la oferta y concentrando los datos en sectores medios y altos. Esto afecta la capacidad de desarrollar modelos generalizables y limita la comprensión del mercado en zonas populares o periféricas.

Este trabajo busca contribuir a esta brecha mediante el desarrollo de un modelo de estimación de precios para apartamentos en Bogotá, integrando variables estructurales con información geoespacial proveniente de capas oficiales (barrios, UPZ, localidades, estratos, avalúos) y puntos de interés de OpenStreetMap. Se evalúa el impacto real de estas variables sobre la precisión y la interpretabilidad del modelo, comparando diferentes algoritmos y analizando la importancia de las características y la distribución del error.

La investigación se estructura en tres componentes principales: (i) construcción de un conjunto de datos reproducible que combine información estructural, administrativa y espacial; (ii) comparación de modelos con y sin variables geográficas mediante validación cruzada; y (iii) análisis interpretativo para identificar los factores que determinan el precio de la vivienda en Bogotá. El documento se organiza en una sección de Metodología que detalla el procesamiento de datos y los modelos empleados; una sección de Resultados donde se reportan las métricas obtenidas; una Discusión que contextualiza los hallazgos en las dinámicas del mercado inmobiliario; y una Conclusión donde se sintetizan los aportes y se proponen líneas futuras de investigación.
