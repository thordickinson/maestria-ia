% En ella, el autor presenta y señala la importancia, el origen (los antecedentes teóricos y prácticos), los objetivos, los alcances, las limitaciones, la metodología empleada, el significado que el estudio tiene en el avance del campo respectivo y su aplicación en el área investigada.

% Entre 60 y 80 páginas

De acuerdo con la NTC1486, la introducción es un espacio dónde el autor presenta y señala la importancia, el origen (los antecedentes teóricos y prácticos), los objetivos, los alcances, las limitaciones, la metodología empleada, el significado que el estudio tiene en el avance del campo respectivo y su aplicación en el área investigada. No debe confundirse con el resumen, ni contener un recuento detallado de la teoría, el método o los resultados, como tampoco anticipar las conclusiones y recomendaciones~\cite{NTC14862008}.\\

Para el desarrollo del documento utilizando la plantilla en \LaTeX se recomienda la guía \textit{the not so short guide to \LaTeX}~\cite{Oetiker2016} que brinda una introducci\'on breve pero muy completa. En donde se presentan entro otras cosas los comandos y ambeintes para el trabajo con gr\'aficas y tablas, como las que se muestran a continuaci\'on:

\begin{figure}[htbp!]
    \caption{Muestra de inclusio\'on de un elemento gr\'afico}
    \centering
    \includegraphics[width=0.8\textwidth]{Images/ImagenYoda.jpg}
    \label{fig:yoda}
\end{figure}

As\'i la Figura~\ref{fig:yoda} muestra una imagen que se encuentra en el directorio images, mientras la tabla~\ref{tab:tablasEx}, muestra dos ejemplos de información tabular con combinaci\'on de columnas, y combinaci\'on de filas.

\begin{table}[htbp!]
    \centering
    \caption{Ejemplo de tabla con multi columnas (arriba) y multi filas(abajo)}
    \begin{tabular}{ |p{3cm}||p{3cm}|p{3cm}|p{3cm}|  }
        \hline
        \multicolumn{4}{|c|}{Country List} \\
        \hline
        Country Name or Area Name& ISO ALPHA 2 Code &ISO ALPHA 3 Code&ISO numeric Code\\
        \hline
        Afghanistan   & AF    &AFG&   004\\
        Aland Islands&   AX  & ALA   &248\\
        Albania &AL & ALB&  008\\
        Algeria    &DZ & DZA&  012\\
        American Samoa&   AS  & ASM&016\\
        Andorra& AD  & AND   &020\\
        Angola& AO  & AGO&024\\
        \hline
    \end{tabular}\\
    \vspace{0.5cm}
    \begin{tabular}{ |c|c|c|c| } 
        \hline
        col1 & col2 & col3 \\
        \hline
        \multirow{3}{4em}{Multiple row} & cell2 & cell3 \\ 
        & cell5 & cell6 \\ 
        & cell8 & cell9 \\ 
        \hline
    \end{tabular}

    \label{tab:tablasEx}
\end{table}

Adem\'as se recomienda consultar los manuales que aparecen en al secci\'on de aprendizaje de overleaf, por ejemplo el de \href{https://www.overleaf.com/learn/latex/Tables}{tablas}.