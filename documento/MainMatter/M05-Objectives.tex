% Incluye los objetivos general y específicos del proyecto. Debe procurar seguir una metodología de formulación de objetivos, por ejemplo S.M.A.R.T.

% Debe tener solo un verbo en infinitivo e incluír como lo van a medir a uno


\section{Objetivo general}

Desarrollar un modelo de predicción de precios de viviendas en Bogotá utilizando técnicas de \textit{machine learning}, que incorpore tanto características internas de las propiedades (como tamaño, número de habitaciones y ubicación) como datos externos (indicadores de seguridad, proximidad a puntos de interés, y accesibilidad a servicios), con el fin de mejorar la precisión en la estimación de precios y contribuir a la toma de decisiones informada en el mercado inmobiliario.


% Preprocesamiento
% Evitar modelos que se van a usar (modelos predictivos)
% No poner métricas

% General un modelo predictivo

\section{Objetivos específicos}
\begin{itemize}
    \item Crear una base de datos consolidada a partir de la recolección de información mediante \textit{web scraping} de diversas plataformas inmobiliarias en Bogotá.
    \item Enriquecer los datos obtenidos integrando información adicional proveniente de fuentes externas, como indicadores de seguridad, convivencia y proximidad a servicios.
    \item Seleccionar el mejor modelo de predicción de precios a partir de la comparación de un conjunto de modelos desarrollados, evaluando su rendimiento mediante métricas de precisión y eficiencia.
\end{itemize}

