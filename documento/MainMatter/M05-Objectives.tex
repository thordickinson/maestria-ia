% Incluye los objetivos general y específicos del proyecto. Debe procurar seguir una metodología de formulación de objetivos, por ejemplo S.M.A.R.T.

% Debe tener solo un verbo en infinitivo e incluír como lo van a medir a uno


\section{Objetivo general}

Desarrollar un sistema de estimación de precios de vivienda en Bogotá, basado en técnicas de aprendizaje automático y datos abiertos, que integre variables estructurales y geoespaciales y que permita a los usuarios acceder a las predicciones a través de una herramienta web con visualización contextual.

\section{Objetivos específicos}
\begin{itemize}
    \item Consolidar un conjunto de datos de viviendas en Bogotá a partir de fuentes públicas y bases de datos disponibles en línea, integrando información estructural (área, número de habitaciones, baños, parqueaderos, estrato) y espacial (ubicación geográfica, barrio, UPZ, proximidad a puntos de interés).
    \item Estandarizar, limpiar e imputar los datos garantizando su coherencia y calidad para el entrenamiento de modelos predictivos.
    \item Desarrollar y comparar distintos algoritmos de aprendizaje automático para la estimación de precios, seleccionando el modelo con mejor desempeño y capacidad explicativa.
    \item Incorporar variables geoespaciales y evaluar su impacto sobre la precisión y la interpretabilidad del modelo.
    \item Diseñar e implementar una utilidad web que permita al usuario ingresar las características de un inmueble y obtener una estimación de precio junto con información estadística contextual (por ejemplo, precios promedio en el sector, accesibilidad y servicios cercanos).
\end{itemize}
