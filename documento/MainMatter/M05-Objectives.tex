% Incluye los objetivos general y específicos del proyecto. Debe procurar seguir una metodología de formulación de objetivos, por ejemplo S.M.A.R.T.

% Debe tener solo un verbo en infinitivo e incluír como lo van a medir a uno


\section{Objetivo general}

Desarrollar un sistema de estimación de precios de vivienda en Bogotá basado en técnicas de aprendizaje automático y datos abiertos, capaz de integrar información estructural y geoespacial para producir predicciones precisas y ofrecerlas mediante una herramienta web con visualización contextual.

\section{Objetivos específicos}
\begin{itemize}
    \item Construir un conjunto de datos de viviendas en Bogotá a partir de fuentes públicas, consolidando información estructural y geoespacial relevante para la estimación de precios.
    \item Garantizar la calidad del conjunto de datos mediante procesos de estandarización, limpieza e imputación adecuados para el entrenamiento de modelos predictivos.
    \item Evaluar diversos algoritmos de aprendizaje automático con el fin de identificar el modelo que ofrezca el mejor desempeño y capacidad explicativa en la predicción de precios.
    \item Analizar el aporte de las variables geoespaciales en la precisión y la interpretabilidad del modelo predictivo.
    \item Implementar una herramienta web que permita a los usuarios ingresar características de un inmueble y acceder a una estimación de precio acompañada de información contextual del entorno urbano.
\end{itemize}
