% describir brevemente que just¿ifica el desarrollo del proyecto de investigación

% incluir el alcance que tendrá el proyecto a desarrollar

La vivienda es un derecho fundamental reconocido en la Constitución de Colombia \cite{ConstitucionPoliticaColombia1991}, pero la inequidad en el acceso, el déficit habitacional y la especulación inmobiliaria muestran que este derecho no se garantiza de manera efectiva. Desarrollar un modelo predictivo robusto, que incorpore variables estructuradas y contextuales, podría ayudar a mitigar los problemas asociados al mercado inmobiliario. Este enfoque no solo permitiría mejorar las estimaciones de precios, sino también diseñar políticas públicas más efectivas y basadas en evidencia \cite{fortaleciendoVivienda2024, castillo2004anotaciones, coyunturaVivienda2023}.

La predicción precisa de precios de viviendas es un aspecto crucial en el mercado inmobiliario, no solo para compradores y vendedores, sino también para inversionistas, desarrolladores y gobiernos locales. Cuando las estimaciones de precios son imprecisas o están basadas en datos incompletos, se generan varios problemas que afectan tanto al mercado como a los actores involucrados.

\subsection*{Decisiones de compra y venta desinformadas}
Los propietarios que subestiman el valor de su vivienda pueden venderla por debajo de su verdadero precio de mercado, perdiendo oportunidades de obtener una mayor ganancia. Por otro lado, los compradores que reciben estimaciones infladas pueden adquirir propiedades por encima de su valor real, enfrentando dificultades para revenderlas o recuperar la inversión. Este fenómeno afecta directamente la equidad y la transparencia del mercado, creando desconfianza entre los participantes.

\subsection*{Desigualdad en la accesibilidad a la vivienda}
Cuando las estimaciones de precios no son precisas, las zonas de alto crecimiento pueden ser sobrevaloradas, y las de menor crecimiento pueden ser infravaloradas, lo que provoca una distorsión en la accesibilidad a la vivienda. Esto puede llevar a que sectores de la población, especialmente los de menores ingresos, sean excluidos de áreas en proceso de valorización, acelerando fenómenos de gentrificación y desplazamiento de comunidades.

\subsection*{Burbujas inmobiliarias y volatilidad en el mercado}
Las malas estimaciones de precios pueden contribuir a la formación de burbujas inmobiliarias, donde los precios se inflan artificialmente debido a una sobrevaloración de las propiedades. Cuando la burbuja estalla, los precios caen bruscamente, lo que provoca una crisis de confianza en el mercado y pérdidas económicas significativas para propietarios e inversionistas. Este tipo de volatilidad afecta la estabilidad financiera de las familias y las inversiones de largo plazo.

\subsection*{Dificultades para la planificación urbana}
Una buena estimación de precios es clave para la planificación urbana y la asignación de recursos en infraestructura, servicios públicos y desarrollo sostenible. Si las predicciones de precios no reflejan adecuadamente el valor futuro de las propiedades, los gobiernos locales y desarrolladores pueden tomar decisiones incorrectas sobre dónde invertir en infraestructura y servicios. Esto puede resultar en zonas sobrepobladas sin servicios adecuados o en áreas subdesarrolladas que no reciben suficiente inversión.

\subsection*{Pérdida de confianza de inversionistas}
Los inversionistas dependen de estimaciones precisas de precios para identificar oportunidades de crecimiento en diferentes zonas. Si los modelos utilizados no logran prever con exactitud la evolución de los precios, los inversionistas podrían enfrentar pérdidas o no alcanzar la rentabilidad esperada. Esto desalienta la inversión en el sector inmobiliario y puede impactar negativamente en el desarrollo económico local.

\subsection*{Impacto en el crédito hipotecario}
Las instituciones financieras basan sus decisiones de otorgamiento de crédito hipotecario en evaluaciones precisas del valor de las propiedades. Cuando estas evaluaciones están basadas en estimaciones incorrectas, el riesgo de impagos o de tener propiedades sobrevaluadas en el portafolio de préstamos aumenta. Esto puede derivar en pérdidas para los bancos y en restricciones más severas para otorgar créditos, afectando a las familias que buscan adquirir vivienda.

La correcta predicción de precios, basada en datos estructurados y enriquecidos, no solo mejorará la transparencia y eficiencia del mercado inmobiliario en Bogotá, sino que también evitará estos problemas derivados. Por lo tanto, el desarrollo de un modelo de predicción que incorpore tanto características internas de las propiedades como datos externos (seguridad, cercanía a servicios, etc.) podría ser fundamental para la sostenibilidad y el crecimiento del mercado inmobiliario, promoviendo una mayor equidad y una mejor toma de decisiones por parte de todos los actores involucrados.
