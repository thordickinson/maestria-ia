El sistema desarrollado fue validado mediante un esquema de validación cruzada de diez pliegues, lo que permitió evaluar su precisión y estabilidad. Se construyeron dos modelos: uno base, con variables estructurales y administrativas, y otro enriquecido reducido, que incorporó información geoespacial derivada de puntos de interés y capas urbanas.

\subsection*{Desempeño del modelo base}

El modelo base, entrenado únicamente con variables internas del inmueble, alcanzó un \textbf{RMSE promedio de 0.151}, un \textbf{MAE de 0.107} y un \textbf{$R^2 = 0.965$}, evidenciando alta capacidad predictiva y estabilidad. Los atributos más influyentes fueron el área construida, el estrato socioeconómico, el número de parqueaderos y la antigüedad (Figura~\ref{fig:importancia_base}).

Este resultado confirma que la estructura física y el contexto socioeconómico del inmueble concentran la mayor parte de la varianza explicativa. En la práctica, refleja el mismo patrón que utilizan las inmobiliarias al fijar precios de oferta: comparaciones locales basadas en el precio por metro cuadrado y ajustes menores por características adicionales. El modelo mantuvo un error relativo estable (3\%–4\%) a lo largo del rango de precios, con mayor desviación en zonas de alta valorización (Chicó, Santa Cecilia, Cerros de Suba), donde factores no registrados como vista o acabados pueden alterar el valor.

\begin{figure}[h]
    \centering
    \includegraphics[width=0.8\linewidth]{Images/feature_importance_base.png}
    \caption{Importancia de variables en el modelo base (XGBoost optimizado).}
    \label{fig:importancia_base}
\end{figure}

\subsection*{Desempeño del modelo enriquecido reducido}

El modelo enriquecido reducido integró variables espaciales derivadas de proximidad a servicios, densidad de puntos de interés y codificación geográfica (geohash). Tras la misma validación cruzada, obtuvo \textbf{RMSE = 0.165}, \textbf{MAE = 0.121} y \textbf{$R^2 = 0.958$}, una ligera degradación frente al modelo base (+9.2\% en RMSE).

Aunque no mejoró la precisión, aportó una lectura más interpretativa del espacio urbano. La \textbf{latitud y longitud} emergieron entre las variables más relevantes, seguidas por área, antigüedad y parqueaderos (Figura~\ref{fig:importancia_reducido}). La permanencia de la variable \texttt{upz} fue un hallazgo clave: este indicador territorial agrupa zonas homogéneas en estrato, morfología y equipamientos, actuando como una unidad óptima de análisis espacial. Su relevancia confirma que la estructura socioespacial de la ciudad está implícitamente codificada en este nivel administrativo.

\begin{figure}[h]
    \centering
    \includegraphics[width=0.8\linewidth]{Images/feature_importance_reducido.png}
    \caption{Importancia de variables en el modelo enriquecido reducido.}
    \label{fig:importancia_reducido}
\end{figure}

\subsection*{Comparación y análisis}

La Tabla~\ref{tab:comparacion_modelos} resume el rendimiento comparativo. El modelo base mantiene la mejor precisión, pero el enriquecido introduce mayor interpretabilidad espacial sin pérdida sustancial de ajuste. Ambos replican fielmente las dinámicas del mercado bogotano, donde el área y la ubicación definen el valor principal del inmueble.

\begin{table}[h]
    \centering
    \caption{Comparación de desempeño entre modelos.}
    \label{tab:comparacion_modelos}
    \begin{tabular}{lcccc}
        \toprule
        \textbf{Modelo} & \textbf{RMSE} & \textbf{MAE} & \textbf{$R^2$} & \textbf{ΔRMSE (\%)} \\
        \midrule
        Base (XGBoost) & 0.151 & 0.107 & 0.965 & 0.0 \\
        Enriquecido reducido & 0.165 & 0.121 & 0.958 & +9.2 \\
        \bottomrule
    \end{tabular}
\end{table}

Las variables geoespaciales mostraron un efecto predictivo limitado, en parte por redundancia con los estratos socioeconómicos y por la homogeneidad de los conteos de servicios en radios fijos. Sin embargo, su incorporación permitió entender la distribución territorial del mercado y visualizar zonas con comportamiento diferenciado.

El análisis de error por sectores confirmó que el modelo generaliza bien en contextos homogéneos, mientras que las zonas de alta valorización presentan mayor incertidumbre asociada a factores cualitativos no registrados. En conjunto, los resultados indican que las variables estructurales y socioeconómicas explican la mayor parte del precio, y que la información geográfica mejora la comprensión del fenómeno sin alterar significativamente la precisión numérica.
