\section{Resultados}
\subsection{Modelo base}
Se evaluaron múltiples algoritmos con validación cruzada (KFold=5, \texttt{shuffle=True}, \texttt{random\_state=42}), usando RMSE en escala real. El mejor desempeño base lo obtuvo \textbf{Random Forest}:
\begin{itemize}
    \item \textbf{CV (aprox)}: RF \(\approx\) 245M; XGB \(\approx\) 245M; LGBM \(\approx\) 246M; lineales \(\approx\) 348M; SVR \(\approx\) 913M.
    \item \textbf{Hold-out (20\%)}: RF RMSE \(\approx\) 250.6M; MAE \(\approx\) 129.9M; R\textsuperscript{2} \(\approx\) 0.915.
\end{itemize}
El uso de transformaciones \(\log(\textit{precio\_venta})\) y \(\log(\textit{area})\) mejoró las métricas al regresar a la escala original.

\subsection{Modelos con datos aumentados}
Tras enriquecer con variables contextuales (POIs por radio, asignación de UPZ/barrio/localidad, avalúos por geohash) se entrenaron variantes XGBoost:
\begin{itemize}
    \item \textbf{v1 (reducido + \textit{barrio\_top})}: RMSE \(\approx\) 254.66M; MAE \(\approx\) 136.54M; R\textsuperscript{2} \(\approx\) 0.9139. Modelo exportado: \texttt{xgboost\_model\_2.1.pkl}.
    \item \textbf{v2 (búsqueda aleatoria)}: hiperparámetros: \(n\_\textit{estimators}=500\), \(\textit{max\_depth}=9\), \(\textit{learning\_rate}=0.05\), \(\textit{subsample}=0.8\), \(\textit{colsample\_bytree}=0.8\), \(\alpha=0\), \(\lambda=1\). Hold-out: RMSE \(\approx\) 233.49M; MAE \(\approx\) 121.96M; R\textsuperscript{2} \(\approx\) 0.9276. Modelo exportado: \texttt{xgboost\_model\_2.2.pkl}.
\end{itemize}

\subsection{Importancia de variables y hallazgos}
\begin{itemize}
    \item \textbf{Variables dominantes}: \emph{área}, \emph{latitud}, \emph{administración}, \emph{antigüedad}, \emph{longitud} aparecen consistentemente entre las más relevantes.
    \item \textbf{Contexto espacial}: las variables de POIs por radio aportan de forma complementaria; su efecto es menor que el de las estructurales pero mejora el ajuste en v2.
    \item \textbf{Distribución del error}: el percentil 80 del error absoluto se ubica en torno a \(\sim\)187M para el RF base; v2 reduce RMSE y MAE manteniendo estabilidad.
\end{itemize}

\subsection{Artefactos y reproducibilidad}
\begin{itemize}
    \item Modelos serializados en \texttt{analisis/data/models/}: \texttt{randomforest\_model\_base.pkl}, \texttt{xgboost\_model\_2.1.pkl}, \texttt{xgboost\_model\_2.2.pkl}.
    \item Dataset enriquecido: \texttt{analisis/data/aptos\_bogota\_enriched.csv}.
    \item Pipeline de estimación expuesto vía \texttt{GET /api/estimate} en el backend (ver metodología).
\end{itemize}

