En esta sección se presentan los resultados obtenidos tras el entrenamiento y evaluación de los 
tres modelos considerados: (i) modelo base, (ii) modelo enriquecido y 
(iii) modelo enriquecido optimizado. Todos los modelos fueron evaluados mediante 
validación cruzada de diez pliegues, registrando los valores individuales de RMSE, MAE y 
R\textsuperscript{2} por pliegue, así como sus estadísticas descriptivas. Además, se 
realizaron pruebas t pareadas para determinar la significancia estadística de las diferencias 
entre modelos.

\subsection{Desempeño del modelo base}

El modelo base, construido exclusivamente con variables estructurales del inmueble, obtuvo un 
\textbf{RMSE promedio de 0.1514} (\(\sigma = 0.0039\)), un \textbf{MAE promedio de 0.1073} 
(\(\sigma = 0.0023\)) y un \textbf{R\textsuperscript{2} promedio de 0.9650} 
(\(\sigma = 0.0020\)). Estos valores indican alta estabilidad y capacidad predictiva 
a lo largo de los diez pliegues.

Las variables más influyentes fueron el área construida, la latitud y la longitud, 
la administración y el número de parqueaderos, como se observa en la 
Figura~\ref{fig:importancia_base}. Estas variables capturan tanto la estructura física 
del inmueble como proxies espaciales y socioeconómicos inherentes al mercado bogotano.

\begin{figure}[h]
    \centering
    \includegraphics[width=0.80\linewidth]{Images/feature_importance_base.png}
    \caption{Importancia de variables en el modelo base (XGBoost optimizado).}
    \label{fig:importancia_base}
\end{figure}

\subsection{Desempeño del modelo enriquecido}

El modelo enriquecido incorporó variables geoespaciales derivadas de OSM, tales como 
densidad de servicios, accesibilidad comercial y puntos de interés en radios de proximidad. 
Este modelo obtuvo un \textbf{RMSE promedio de 0.1705} y un \textbf{MAE de 0.1260}, 
con \textbf{R\textsuperscript{2} = 0.9560}. Aunque la inclusión de variables espaciales 
incrementó la interpretabilidad del modelo, produjo un aumento significativo en el error 
respecto al modelo base.

La Figura~\ref{fig:importancia_enriquecido} muestra la importancia relativa de las 
variables en este modelo. Los patrones observados indican que la ubicación geográfica 
(latitud y longitud) concentra casi todo el aporte espacial efectivo, mientras que los 
conteos de servicios añadidos presentan valores marginales.

\begin{figure}[h]
    \centering
    \includegraphics[width=0.80\linewidth]{Images/feature_importance_enriquecido.png}
    \caption{Importancia de variables en el modelo enriquecido.}
    \label{fig:importancia_enriquecido}
\end{figure}

\subsection{Interpretabilidad mediante SHAP}

Para analizar el aporte global de las características del modelo enriquecido, se calcularon 
valores SHAP sobre el conjunto transformado. La Figura~\ref{fig:shap_summary} muestra el 
\textit{summary plot}, mientras que la Figura~\ref{fig:shap_bar} presenta el ranking global 
de importancia media absoluta.

Estos resultados permiten identificar las variables con mayor contribución y justificar la 
selección del subconjunto utilizado en el modelo enriquecido optimizado.

\begin{figure}[h]
    \centering
    \includegraphics[width=0.80\linewidth]{Images/shap_summary.png}
    \caption{SHAP summary plot del modelo enriquecido.}
    \label{fig:shap_summary}
\end{figure}

\begin{figure}[h]
    \centering
    \includegraphics[width=0.80\linewidth]{Images/shap_barplot.png}
    \caption{Importancia media absoluta de los valores SHAP.}
    \label{fig:shap_bar}
\end{figure}

\subsection{Desempeño del modelo enriquecido optimizado}

A partir del ordenamiento global de variables según SHAP, se seleccionó un subconjunto de 
características con mayor contribución. El modelo entrenado con este subconjunto obtuvo un 
\textbf{RMSE promedio de 0.1584} (\(\sigma = 0.0082\)), \textbf{MAE de 0.1138} 
(\(\sigma = 0.0047\)) y \textbf{R\textsuperscript{2} = 0.9600} 
(\(\sigma = 0.0059\)). Sus resultados se encuentran entre los del modelo base y los del modelo 
enriquecido, confirmando que una selección explícita de variables reduce ruido sin mejorar la 
precisión global.

Las variables seleccionadas se listan en la Tabla~\ref{tab:variables_optimizadas}.

\begin{table}[h]
    \centering
    \caption{Variables seleccionadas para el modelo enriquecido optimizado.}
    \label{tab:variables_optimizadas}
    \begin{tabular}{ll}
        \toprule
        \textbf{Variable} & \textbf{Importancia SHAP media} \\
        \midrule
        área & 0.374 \\
        latitud & 0.099 \\
        parqueaderos & 0.084 \\
        baños & 0.070 \\
        longitud & 0.064 \\
        administración & 0.064 \\
        antigüedad & 0.046--0.042 \\
        gimnasio & 0.020 \\
        habitaciones & 0.013 \\
        ascensor & 0.012 \\
        piscina & 0.011 \\
        \bottomrule
    \end{tabular}
\end{table}

\subsection{Comparación cuantitativa y pruebas estadísticas}

La Tabla~\ref{tab:comparacion_modelos_final} resume el desempeño comparado de los tres 
modelos. El modelo base obtiene la mejor precisión, seguido por el modelo enriquecido optimizado, 
mientras que el modelo enriquecido presenta el peor desempeño.

\begin{table}[h]
    \centering
    \caption{Comparación de desempeño entre modelos (10-Fold CV).}
    \label{tab:comparacion_modelos_final}
    \begin{tabular}{lccc}
        \toprule
        \textbf{Modelo} & \textbf{RMSE} & \textbf{MAE} & \textbf{$R^2$} \\
        \midrule
        Base & 0.1514 $\pm$ 0.0039 & 0.1073 $\pm$ 0.0023 & 0.9650 $\pm$ 0.0020 \\
        Enriquecido & 0.1705 $\pm$ 0.0050 & 0.1260 $\pm$ 0.0030 & 0.9560 $\pm$ 0.0030 \\
        Enriquecido Optimizado & 0.1584 $\pm$ 0.0082 & 0.1138 $\pm$ 0.0047 & 0.9600 $\pm$ 0.0059 \\
        \bottomrule
    \end{tabular}
\end{table}

Las pruebas t pareadas realizadas con los diez valores de RMSE para cada modelo arrojaron:

\begin{itemize}
    \item Base vs.\ Enriquecido: \(p = 5.49 \times 10^{-6}\)
    \item Base vs.\ Enriquecido Optimizado: \(p = 0.0112\)
    \item Enriquecido vs.\ Enriquecido Optimizado: \(p = 0.0037\)
\end{itemize}

Estos resultados indican que todas las diferencias observadas entre modelos son 
estadísticamente significativas. En particular, el modelo base es estadísticamente 
superior a los otros dos, y el modelo enriquecido optimizado supera al modelo enriquecido completo.

