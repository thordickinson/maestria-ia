Los resultados del estudio muestran que la estimación del precio de apartamentos en Bogotá puede lograrse con alta precisión utilizando principalmente dos tipos de información: el área del inmueble y su ubicación. El modelo base, construido únicamente con variables estructurales, alcanzó el mejor desempeño con un RMSE promedio de 0.1514 y un $R^2$ de 0.965, superando estadísticamente a los modelos enriquecidos según las pruebas t realizadas. Esto indica que, más allá de características adicionales del edificio o variables derivadas, la combinación de tamaño y localización concentra la mayor parte de la señal predictiva.

Las variables geoespaciales añadidas no mejoraron la precisión del modelo, en parte porque la ubicación ya incorpora implícitamente factores socioeconómicos, disponibilidad de servicios y condiciones del entorno urbano. La relevancia observada en variables como latitud, longitud y UPZ confirma que la estructura territorial de la ciudad está fuertemente codificada en la localización misma, haciendo redundantes muchos indicadores espaciales derivados. El modelo enriquecido optimizado retuvo principalmente atributos estructurales, reforzando esta misma conclusión.

Estos hallazgos sugieren que mejoras futuras deben orientarse hacia mayor granularidad espacial, tiempos reales de accesibilidad y modelos especializados por zonas, más que hacia la expansión de variables espaciales agregadas. El área y la ubicación siguen siendo los determinantes principales del precio en el mercado inmobiliario bogotano.
