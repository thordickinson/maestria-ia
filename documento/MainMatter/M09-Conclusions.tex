El presente estudio permitió desarrollar y evaluar un modelo de estimación de precios de vivienda en Bogotá utilizando técnicas de aprendizaje automático y datos abiertos. El enfoque comparó un modelo base —con variables estructurales y socioeconómicas— con un modelo enriquecido que incorporó información geoespacial derivada de puntos de interés y divisiones urbanas.  

Los resultados demuestran que las características físicas del inmueble, en particular el área construida, el número de parqueaderos y la antigüedad, junto con el estrato socioeconómico, explican la mayor parte de la variabilidad del precio. Estos factores no solo reflejan condiciones internas del bien, sino también la calidad y ubicación del entorno en que se encuentra. En este sentido, el estrato y la localización operan como variables proxy del acceso a servicios, infraestructura y valorización territorial, por lo que la ubicación emerge como el determinante subyacente del valor inmobiliario.

El modelo base alcanzó un rendimiento sobresaliente ($R^2 = 0.965$, RMSE = 0.151), mientras que el modelo enriquecido redujo ligeramente la precisión pero aportó mayor interpretabilidad espacial. La permanencia de la variable \texttt{upz} evidencia que las divisiones urbanas oficiales capturan de forma robusta la estructura socioespacial de la ciudad y sintetizan múltiples dimensiones del entorno.  

En consecuencia, se concluye que el precio de la vivienda en Bogotá depende fundamentalmente del área y de las variables socioeconómicas asociadas al estrato, las cuales representan indirectamente la localización del inmueble. El valor de las variables geoespaciales no radica en mejorar las métricas de error, sino en profundizar la comprensión del territorio y de los patrones de mercado que lo configuran.  

A futuro, se propone avanzar hacia modelos sectorizados o de mayor granularidad espacial que permitan capturar las particularidades de cada zona urbana. Estos enfoques podrían fortalecer la capacidad predictiva y ofrecer herramientas más precisas para la gestión del suelo, la planificación urbana y la toma de decisiones en el mercado inmobiliario.
