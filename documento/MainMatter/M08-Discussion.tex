El propósito central del estudio fue evaluar si el enriquecimiento geoespacial mejoraba la predicción de precios de vivienda en Bogotá frente a un modelo basado únicamente en variables estructurales. Los resultados muestran que, aunque las variables espaciales no incrementaron sustancialmente la precisión, sí aportaron valor interpretativo al permitir visualizar cómo la estructura territorial influye en la formación del precio.

El modelo base, entrenado con características físicas y administrativas del inmueble, obtuvo el mejor desempeño entre los tres modelos comparados. Este resultado confirma que variables como área, parqueaderos, administración y antigüedad concentran la mayor parte de la varianza explicativa. La fuerte correlación entre estructura física y ubicación en Bogotá explica este comportamiento: en la práctica, los precios se estiman a partir del \textit{precio por metro cuadrado} y comparaciones locales, por lo que buena parte de la información espacial ya se encuentra implícita en las variables tradicionales de tasación.

Aunque el modelo enriquecido incorporó conteos de servicios, accesibilidad y agregaciones espaciales, su efecto predictivo fue limitado. La redundancia espacial entre inmuebles cercanos, la segmentación socioeconómica marcada por el estrato y la resolución espacial relativamente gruesa explican por qué muchas variables añadidas no aportaron señal adicional. Los valores SHAP mostraron que la mayor parte del aporte geográfico se concentra en latitud y longitud, mientras que la mayoría de POIs incrementa el ruido más que la precisión.

La relevancia persistente de la UPZ dentro del modelo enriquecido también es coherente con la estructura urbana de la ciudad. Estas unidades territoriales agrupan zonas con características socioeconómicas y morfológicas homogéneas, por lo que funcionan como una división espacial intermedia capaz de sintetizar gran parte del contexto urbano relevante.

Los resultados deben interpretarse considerando los sesgos de los datos. Los portales inmobiliarios priorizan la publicación de inmuebles con mayor valor comercial, lo que sobrerrepresenta zonas de estrato medio y alto y subrepresenta áreas periféricas. En estratos bajos, además, las transacciones suelen realizarse por fuera de canales formales, lo que limita su presencia en bases públicas. Este patrón explica por qué el modelo presenta menor error en zonas de alta valorización y mayor variabilidad en sectores con menos información disponible.

Durante la selección de variables, características que intuitivamente podrían ser importantes —como número de habitaciones o presencia de ascensor— no fueron retenidas por su bajo aporte marginal. En Bogotá, el área total suele capturar el efecto del número de habitaciones, mientras que el impacto del ascensor depende del piso del inmueble, variable no incluida en el conjunto de datos.

El análisis del error evidencia que el modelo generaliza bien en zonas homogéneas, pero presenta desviaciones mayores en áreas con atributos cualitativos no registrados, como vista, acabados o reputación del edificio. Esto sugiere que futuros modelos podrían beneficiarse de información más granular sobre características internas y de conectividad.

Aunque el enriquecimiento geoespacial no mejora la precisión, sí amplía la capacidad interpretativa del sistema. Con base en los hallazgos, se identifican varias líneas de trabajo futuro: incorporar capas más detalladas (manzana o código catastral), incluir medidas de accesibilidad basadas en tiempos reales de viaje, integrar variables económicas dinámicas, explorar modelos espaciales avanzados y entrenar modelos específicos por zonas o localidades para reducir la heterogeneidad territorial.

En conjunto, los resultados muestran que la estructura del mercado inmobiliario bogotano está fuertemente determinada por variables físicas y socioeconómicas del inmueble. Las variables de entorno aportan profundidad analítica y contextual, pero no modifican de manera sustancial la precisión numérica del modelo predictivo.
