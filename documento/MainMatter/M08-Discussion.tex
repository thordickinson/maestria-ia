El presente estudio tuvo como propósito evaluar el impacto del enriquecimiento geoespacial en la predicción de precios de vivienda en Bogotá, analizando si la incorporación de variables contextuales (proximidad a servicios, equipamientos urbanos y características de entorno) mejoraba la capacidad explicativa de un modelo de aprendizaje automático respecto a uno construido únicamente con variables estructurales. Los resultados mostraron que, aunque las variables espaciales no incrementaron de forma sustancial la precisión del modelo, sí aportaron información valiosa para interpretar la lógica territorial del mercado inmobiliario.

El modelo base, entrenado con variables estructurales y administrativas, presentó un desempeño destacado y estable, con un RMSE promedio de 0.151 y un $R^2$ de 0.965. Este resultado confirma que los atributos físicos del inmueble —principalmente el área, el estrato, el número de parqueaderos y la antigüedad— concentran la mayor parte de la varianza explicativa del precio. La validación cruzada de diez pliegues evidenció una baja dispersión en los errores, lo cual indica robustez y ausencia de sobreajuste. Asimismo, la transformación logarítmica aplicada al precio y al área permitió estabilizar la varianza y mejorar la linealidad de las relaciones, contribuyendo a la calidad del ajuste.

El comportamiento del modelo base refleja, en gran medida, las dinámicas reales de valoración utilizadas por agentes inmobiliarios y portales especializados. En la práctica, las inmobiliarias y asesores comerciales suelen estimar los precios de publicación a partir del \textit{precio por metro cuadrado} de propiedades comparables en la misma zona, ajustando ligeramente por características como número de baños o estado de conservación. Este procedimiento, basado en comparaciones locales, implica que las variables estructurales y socioeconómicas ya contienen implícitamente gran parte de la información espacial que el modelo podría aprender.

Desde esta perspectiva, el alto rendimiento del modelo base no solo valida su solidez técnica, sino que también revela la racionalidad del mercado: la ubicación y el tamaño explican la mayor parte del precio porque reflejan las prácticas habituales de valoración en el sector inmobiliario.

\subsection*{Aporte y limitaciones de las variables geoespaciales}

Aunque las variables geoespaciales no produjeron una mejora significativa en el RMSE, su inclusión permitió una interpretación más rica del entorno urbano. Sin embargo, el efecto predictivo limitado puede explicarse por varios factores:

\begin{itemize}
    \item \textbf{Redundancia espacial:} los inmuebles cercanos tienden a compartir características similares de accesibilidad y entorno. Al utilizar radios fijos (100 a 2000 metros) para el cálculo de puntos de interés, muchos registros presentaron conteos de POIs prácticamente idénticos, lo que redujo la variabilidad informativa.
    \item \textbf{Segmentación del mercado:} los estratos socioeconómicos en Bogotá ya actúan como proxy espacial del acceso a servicios y equipamientos, haciendo que las variables de entorno sean parcialmente redundantes.
    \item \textbf{Resolución de los datos:} la agregación por geohash de nivel 7 (aprox. 150 metros) ofrece una granularidad adecuada, pero puede no captar microdiferencias relevantes en áreas con alta densidad de servicios o variabilidad de precios.
\end{itemize}

A pesar de estas limitaciones, las variables geoespaciales aportaron valor interpretativo. La importancia asignada a la latitud y longitud dentro del modelo reducido sugiere que la localización sigue siendo un componente estructural del valor, aunque su efecto marginal sea menor una vez controladas las variables internas del inmueble.

Un hallazgo destacable del modelo reducido es la permanencia de la variable \texttt{upz}, lo cual resulta coherente con el diseño y la función de las Unidades de Planeamiento Zonal (UPZ) en la estructura urbana de Bogotá. Las UPZ constituyen divisiones territoriales definidas por el Distrito Capital que agrupan sectores con características socioeconómicas, morfológicas y de equipamiento relativamente homogéneas. 

La prevalencia de esta variable en el modelo sugiere que la \textbf{UPZ actúa como un nivel intermedio de agregación espacial} capaz de sintetizar información contextual que de otro modo requeriría múltiples variables geoespaciales. En efecto, los límites de las UPZ tienden a coincidir con zonas de uso del suelo, densidad y estrato similares, lo que las convierte en un indicador robusto del entorno urbano y de su valoración relativa dentro del mercado inmobiliario.

Este resultado refuerza la idea de que la estructura socioespacial de la ciudad se encuentra implícitamente codificada en la variable \texttt{upz}, y que su incorporación permite al modelo capturar las diferencias territoriales sin necesidad de un número excesivo de variables de proximidad o conteo de servicios. Desde un punto de vista práctico, la UPZ puede considerarse una \textbf{unidad de análisis óptima} para la modelación inmobiliaria en Bogotá, al balancear granularidad espacial, estabilidad administrativa y coherencia socioeconómica.

\subsection*{Sesgos y representatividad de los datos}

Un aspecto relevante para comprender los resultados es la naturaleza de los datos utilizados. Los portales inmobiliarios operan bajo esquemas de pago por publicación: las inmobiliarias adquieren paquetes que les permiten listar un número limitado de inmuebles por mes. En consecuencia, tienden a priorizar la publicación de propiedades de mayor valor, que generan comisiones más altas y justifican mejor la inversión publicitaria. Este comportamiento introduce un sesgo estructural en la muestra, sobrerrepresentando inmuebles de estratos medios y altos, y subrepresentando zonas periféricas o de menor valor.

Adicionalmente, en los estratos más bajos de la ciudad es frecuente que las transacciones inmobiliarias se realicen de forma directa entre comprador y vendedor, sin la intervención de agentes o empresas inmobiliarias. Esta práctica responde tanto a la informalidad del mercado como a la necesidad de evitar el pago de comisiones, lo que excluye del registro digital una proporción significativa de operaciones reales. En consecuencia, los portales inmobiliarios —cuyos principales clientes son precisamente las agencias y corredores— reflejan de manera más completa la oferta de vivienda en sectores de estrato 3 en adelante, pero no en los niveles socioeconómicos inferiores.

Este sesgo puede explicar por qué los modelos —tanto el base como el enriquecido— alcanzan un rendimiento mayor en los sectores de alta valorización y presentan errores más altos en áreas con menor presencia de datos. En otras palabras, el modelo no solo refleja patrones del mercado inmobiliario, sino también las dinámicas de visibilidad y acceso a la información dentro de los portales digitales, condicionadas por la estructura comercial del sector.

Durante el proceso de optimización y reducción de características, algunas variables que intuitivamente podrían considerarse relevantes fueron excluidas del modelo final por presentar una contribución estadísticamente marginal. Entre ellas destacan el número de habitaciones (\texttt{habitaciones}) y la presencia de ascensor (\texttt{ascensor}). 

La ausencia de la variable \texttt{habitaciones} puede interpretarse en el contexto del mercado inmobiliario local, donde el \textbf{área total} actúa como principal indicador del tamaño y, por tanto, captura indirectamente la influencia del número de habitaciones. En los portales inmobiliarios y en la práctica de tasación informal, los compradores y agentes suelen evaluar el valor de un inmueble principalmente con base en el precio por metro cuadrado, lo que reduce la sensibilidad del modelo a la cantidad de habitaciones.

De forma similar, la variable \texttt{ascensor}, aunque puede tener un impacto perceptible en el valor de los pisos altos o bajos, no mostró una importancia significativa en el modelo reducido. Esto podría deberse a que su efecto se diluye dentro del conjunto de características estructurales y de localización, o bien a que la relación entre el precio y la altura del piso no se encuentra suficientemente representada en los datos disponibles. En edificaciones sin ascensor, por ejemplo, el valor tiende a concentrarse en los pisos inferiores, mientras que en aquellas que sí cuentan con ascensor, el precio se incrementa en los pisos superiores, pero esta variabilidad no se refleja en la información disponible.

Estas observaciones sugieren que el modelo replica las prácticas reales del mercado, donde la valoración de los inmuebles se concentra en métricas simples como el área o el estrato, dejando de lado factores arquitectónicos o de confort que, aunque relevantes, no siempre están explícitamente incorporados en las bases de datos disponibles.

Otro aspecto relevante observado durante el análisis del modelo reducido es que ninguna de las variables añadidas durante el proceso de enriquecimiento geoespacial fue finalmente retenida en el conjunto de predictores seleccionados. A excepción de la variable \texttt{barrio}, el modelo reducido se compone exclusivamente de atributos estructurales y administrativos del inmueble. Este resultado sugiere que la información espacial incorporada mediante los conteos de puntos de interés (POIs) o las variables agregadas por geohash podría encontrarse \textbf{implícitamente representada} en las características originales de los inmuebles o en el contexto socioeconómico de cada barrio.

En otras palabras, variables como la accesibilidad, la proximidad a servicios o la densidad de equipamientos urbanos pueden estar ya codificadas de forma indirecta a través de variables correlacionadas, como el estrato o la localización del barrio. Este fenómeno es consistente con la estructura espacial del mercado inmobiliario en Bogotá, donde las zonas de mayor valorización tienden a concentrar tanto los servicios urbanos como los precios más altos, generando colinealidad entre variables estructurales y espaciales.

Como línea de trabajo futuro, se plantea la posibilidad de desarrollar \textbf{modelos especializados por sector o localidad}, que permitan capturar las particularidades de cada zona y reducir el error predictivo en áreas con comportamientos de mercado más heterogéneos. Esta estrategia podría mejorar la precisión global del sistema, al tiempo que permitiría identificar factores diferenciales de valorización entre submercados urbanos específicos.

\subsection*{Análisis de error y generalización}

El análisis de error mostró que el modelo mantiene un comportamiento estable a lo largo de los diferentes rangos de precios: el error relativo promedio se mantiene entre 3\% y 4\%. Las desviaciones mayores se concentran en zonas de alta valorización (Chicó, Santa Cecilia, Cerros de Suba), donde intervienen factores cualitativos no registrados, como acabados, vista panorámica o reputación del edificio. En contraste, los sectores intermedios y periféricos presentaron un error más uniforme, lo que sugiere que el modelo generaliza bien en contextos homogéneos.

Este patrón refuerza la idea de que la mayor parte de la varianza explicativa proviene de características estructurales y de mercado, mientras que las variables de entorno aportan información secundaria más útil para análisis espaciales que para predicción directa.

Los resultados demuestran que los modelos basados en aprendizaje automático pueden replicar con alta fidelidad las dinámicas de valoración observadas en el mercado inmobiliario. El enriquecimiento geoespacial, aunque no mejora de forma significativa las métricas de precisión, amplía la interpretabilidad del modelo y permite visualizar los factores contextuales que influyen en la formación de precios. 

De cara a futuras investigaciones, se sugiere:
\begin{itemize}
    \item Incrementar la granularidad espacial (por manzana o código catastral) y temporal (series trimestrales o mensuales).
    \item Incorporar variables de conectividad real, como tiempos de desplazamiento o accesibilidad multimodal.
    \item Integrar fuentes complementarias de datos económicos (tasa de valorización, dinámica de oferta y demanda).
    \item Explorar modelos híbridos o espaciales avanzados (GWR, \textit{graph neural networks}) para capturar autocorrelaciones geográficas.
    \item Desarrollar modelos específicos por sector o localidad que permitan capturar patrones propios de cada submercado urbano.
\end{itemize}

En síntesis, los resultados reflejan tanto la estructura técnica del modelo como las lógicas de mercado que determinan el precio de la vivienda en Bogotá: el área y la ubicación siguen siendo los principales determinantes del valor, mientras que el contexto geográfico aporta profundidad analítica sin alterar de manera sustancial la precisión global del sistema.