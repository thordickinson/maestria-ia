Los resultados confirman que las variables estructurales y socioeconómicas explican la mayor parte del valor de la vivienda en Bogotá, mientras que las variables geoespaciales aportan principalmente un valor interpretativo. El alto desempeño del modelo base evidencia que los factores tradicionales de tasación —como área, estrato y parqueaderos— capturan gran parte de la información espacial implícita en el mercado. Este patrón refleja la lógica del sector inmobiliario local, donde el \textit{precio por metro cuadrado} y las comparaciones dentro de cada zona determinan la mayoría de las decisiones de valoración.

Aunque el enriquecimiento geoespacial no mejoró la precisión global, permitió una comprensión más profunda del contexto urbano. La persistencia de la variable \texttt{upz} sugiere que las divisiones territoriales oficiales sintetizan de manera efectiva la estructura socioespacial de la ciudad, funcionando como un indicador robusto del entorno. En conjunto, los hallazgos muestran que los modelos de aprendizaje automático pueden reproducir con alta fidelidad las prácticas reales del mercado, y que el valor analítico de las variables espaciales radica más en su capacidad de explicar patrones territoriales que en reducir el error predictivo.
