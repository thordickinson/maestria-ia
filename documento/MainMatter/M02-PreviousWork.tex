El análisis y la estimación del valor de los bienes raíces se ha consolidado en la última década como un componente estratégico en la gestión del mercado inmobiliario. La expansión de los portales digitales, el acceso a fuentes de información pública y el auge de la analítica de datos han impulsado el desarrollo de modelos que buscan determinar precios de vivienda de forma transparente, replicable y sustentada en evidencia cuantitativa, en contraste con los métodos tradicionales basados en juicios subjetivos o comparaciones locales. La revisión de la literatura especializada permite identificar cuatro grandes líneas de investigación, diferenciadas por su enfoque metodológico y su contribución al negocio inmobiliario.

Los primeros estudios aplicaron métodos estadísticos convencionales, como la regresión lineal múltiple o los modelos hedónicos, para explicar cómo las características físicas —área, número de habitaciones, baños, garajes o antigüedad— influyen en el valor de una vivienda. Investigaciones como las de \cite{yu2016realestate} y \cite{kim2018machinelearning} evidencian la utilidad de estos modelos en contextos homogéneos, donde la estructura del inmueble explica buena parte de la variabilidad del precio. No obstante, su capacidad predictiva disminuye en mercados urbanos complejos, donde factores como la ubicación o el entorno socioeconómico tienen un peso determinante.

Con el propósito de anticipar fluctuaciones de precios y evaluar impactos derivados de la coyuntura económica, autores como \cite{li2017realestate} y \cite{zhang2018realestate} incorporaron indicadores macroeconómicos —PIB, inflación, tasas hipotecarias, inversión en construcción o empleo— dentro de modelos predictivos. En estos trabajos se emplean técnicas de Support Vector Regression (SVR) y variantes optimizadas mediante Particle Swarm Optimization (PSO), orientadas a capturar relaciones no lineales y generar proyecciones útiles para la planeación estratégica del sector inmobiliario.

El crecimiento de los portales inmobiliarios y la disponibilidad de datos estructurados a gran escala promovieron la adopción de técnicas más avanzadas de aprendizaje automático. \cite{park2015housing} introducen algoritmos de clasificación como C4.5, RIPPER, Naïve Bayes y AdaBoost para analizar datos de múltiples fuentes (MLS, tasas hipotecarias, calidad educativa), destacando su potencial para apoyar la toma de decisiones de agentes y compradores. De manera complementaria, \cite{bigdata2019realestate} exploraron el uso de Random Forest, LASSO y Gradient Boosting sobre conjuntos de datos con decenas de variables explicativas, evidenciando el papel del aprendizaje automático en la automatización de la valoración masiva de propiedades y en la reducción de la subjetividad en las tasaciones.

La integración de arquitecturas neuronales profundas y técnicas de reducción de dimensionalidad ha potenciado la capacidad de los modelos para representar patrones complejos entre variables estructurales, económicas y espaciales. \cite{mostofi2022realestate} proponen un enfoque híbrido PCA–DNN para mejorar la predicción de precios en mercados heterogéneos, mientras que \cite{dabreo2021realestate} aplican redes neuronales y métodos de vecinos más cercanos para estimar precios urbanos en tiempo real. Estas aproximaciones marcan una transición desde modelos descriptivos hacia sistemas predictivos aplicables a portales inmobiliarios, procesos de tasación automatizada y gestión de carteras de inversión.

En el contexto colombiano, \cite{PerezRave2020ApartmentPricesColombia} modelan el precio de apartamentos ofertados en línea para Medellín, Envigado y Sabaneta usando más de 15\,000 anuncios con variables estructurales, de entorno y atributos derivados de texto. Comparan regresión lineal, árboles de decisión, Random Forest y esquemas de \textit{bagging}, encontrando desempeños cercanos a $R^{2} \approx 0{,}99$ y mostrando que área construida, número de baños, estrato y el precio medio por metro cuadrado de la zona concentran la mayor parte del poder explicativo. De forma similar, la tesis de \cite{MedinaGiraldo2023PrediccionPreciosMedellin} construye un modelo de valoración automática para viviendas en Medellín y el Área Metropolitana a partir de anuncios del portal Properati, comparando regresión lineal, árboles de decisión, Random Forest, $k$-NN y XGBoost; el mejor desempeño se obtiene con Random Forest, con $R^{2}\approx 0{,}81$ y un MAPE cercano al 14\,\%, proponiéndose como herramienta operativa para usuarios finales, portales e instituciones fiscales.

A nivel latinoamericano, \cite{Tapia2025AVMChileLightGBM} comparan un modelo hedónico espacial SAR con un AVM basado en LightGBM para la Región Metropolitana de Santiago de Chile, incorporando atributos estructurales, variables de entorno y \textit{features} visuales extraídos mediante CNN. El estudio concluye que el modelo de aprendizaje automático con variables de imagen mejora de forma notable la precisión frente al modelo hedónico, y mediante técnicas de interpretabilidad evidencia interacciones no lineales entre atributos físicos, calidad de colegios, accesibilidad y características del barrio, reforzando la relevancia de integrar información espacial y contextual en los sistemas de valoración.

La literatura revisada muestra una evolución desde modelos explicativos de carácter estadístico hacia modelos predictivos orientados a la gestión y la toma de decisiones empresariales en el mercado inmobiliario. Los avances recientes apuntan a combinar precisión, interpretabilidad y escalabilidad, y ponen de relieve la importancia de considerar variables espaciales y de entorno en mercados urbanos latinoamericanos. En particular, los trabajos desarrollados en Colombia y Chile \cite{PerezRave2020ApartmentPricesColombia,MedinaGiraldo2023PrediccionPreciosMedellin,Tapia2025AVMChileLightGBM} evidencian el potencial de los modelos de aprendizaje automático para capturar la heterogeneidad espacial y soportar aplicaciones prácticas de tasación masiva. En la Tabla~\ref{tab:state_of_art} se resumen algunos de los trabajos más representativos, destacando sus contextos de aplicación, los modelos utilizados y las métricas de evaluación reportadas.

\begin{table}[H]
\centering
\caption{Estudios representativos sobre predicción de precios inmobiliarios}
\label{tab:state_of_art}
\begin{tabular}{p{2cm} p{6cm} p{3.2cm} p{2.1cm}}
\hline
\textbf{Estudio} & \textbf{Contexto} & \textbf{Modelos} & \textbf{Métricas} \\ \hline
\cite{li2017realestate} & Mercado chino; predicción de variaciones de precios basada en indicadores macroeconómicos. & Regresión múltiple, ANN, SVM & RMSE, $R^{2}$ \\
\cite{zhang2018realestate} & Predicción de precios mediante optimización de hiperparámetros con PSO. & PSO–SVM, SVM, BP & MAPE, error relativo \\
\cite{kim2018machinelearning} & Comparación entre modelos hedónicos, de ML y \textit{deep learning}. & Hedonic regression, ANN, CNN & RMSE, MAE, $R^{2}$ \\
\cite{bigdata2019realestate} & Ames (Iowa); análisis de datos masivos y selección de variables relevantes. & Linear Regression, LASSO, Random Forest, Gradient Boosting & RMSE, MAE, $R^{2}$ \\
\cite{wang2019svr} & Predicción del precio inmobiliario mediante SVR. & SVR, BPNN & MAE, MAPE, RMSE \\
\cite{yu2016realestate} & Modelado de precios residenciales usando regresión y clasificación. & Linear Regression, Decision Tree, Random Forest, SVM & RMSE, MAE, Accuracy \\
\cite{dabreo2021realestate} & Aplicación práctica de ML para precios en entornos urbanos. & Linear Regression, KNN, Random Forest, SVM & RMSE, MAE \\
\cite{mostofi2022realestate} & Mercado turco; predicción con \textit{deep learning} y reducción de dimensionalidad. & DNN, SRA-DNN, PCA-DNN & MSE, MAE, MAPE \\
\cite{PerezRave2020ApartmentPricesColombia} & Medellín, Envigado y Sabaneta (Colombia); anuncios de apartamentos en portales web. & Linear Regression, árboles, Random Forest, Bagging & $R^{2}$, RMSE \\
\cite{MedinaGiraldo2023PrediccionPreciosMedellin} & Medellín y Área Metropolitana; datos de Properati. & Linear Regression, Decision Tree, Random Forest, KNN, XGBoost & $R^{2}$, MAPE \\
\cite{Tapia2025AVMChileLightGBM} & Región Metropolitana de Santiago (Chile); AVM con variables espaciales y de imagen. & LightGBM, modelo hedónico SAR & RMSE, MAE, $R^{2}$ \\
\hline
\end{tabular}
\end{table}

\noindent
La revisión presentada permite establecer que la investigación internacional ha logrado avances considerables en precisión y capacidad predictiva, aunque persisten desafíos en términos de acceso a datos, reproducibilidad y adaptabilidad a contextos locales. A partir de estos antecedentes, la presente investigación se enfoca en desarrollar un modelo predictivo aplicable al mercado de apartamentos en Bogotá, utilizando datos abiertos y variables espaciales derivadas de fuentes como OpenStreetMap y el portal de Datos Abiertos de la Alcaldía de Bogotá, lo cual se detalla en la siguiente sección metodológica.
