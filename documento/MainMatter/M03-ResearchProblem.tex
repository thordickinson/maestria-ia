El déficit habitacional en Colombia, caracterizado por la falta de acceso a vivienda digna y equitativa, es una problemática compleja que afecta a millones de personas. Según Castillo (2004), el déficit habitacional en el país incluye tanto aspectos cuantitativos, como la insuficiencia de viviendas disponibles, como cualitativos, relacionados con la calidad de las viviendas y su entorno \cite{castillo2004anotaciones}. Este problema es especialmente agudo en Bogotá, donde el crecimiento desorganizado y la urbanización informal han exacerbado la marginalidad y la exclusión \cite{castillo2004anotaciones}.

Adicionalmente, el acceso limitado al financiamiento hipotecario y la especulación inmobiliaria contribuyen a distorsionar el mercado, encareciendo los precios y excluyendo a los hogares de bajos ingresos. Más del 80\% de los hogares de bajos ingresos no tiene acceso a créditos hipotecarios, lo que agrava la inequidad en el acceso a la vivienda \cite{castillo2004anotaciones}. Este panorama se ve reflejado en los cambios recientes en las políticas de subsidios, como \textit{Mi Casa Ya}, cuya implementación irregular ha afectado tanto a desarrolladores como a compradores, aumentando la incertidumbre en el sector \cite{coyunturaVivienda2023}.

En paralelo, la vivienda también ha sido utilizada históricamente como un instrumento de especulación financiera y política. La falta de integración de datos cualitativos, como la calidad de los servicios públicos, la accesibilidad y el equipamiento urbano, limita la capacidad de los modelos actuales de predicción de precios para reflejar las dinámicas reales del mercado. Los modelos tradicionales, basados únicamente en características extraídas de portales inmobiliarios, no consideran los factores sociales y económicos que afectan directamente los precios, como destacó Walter (2023) en su análisis de la desaceleración del sector vivienda en Colombia \cite{coyunturaVivienda2023, fortaleciendoVivienda2024}.
