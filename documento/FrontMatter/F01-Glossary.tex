% Lista alfabética de términos y sus definiciones o explicaciones necesarios para la comprensión del documento. La existencia de un glosario no justifica la omisión de una explicación en el texto la primera vez que aparece un término. El título glosario se escribe en mayúscula sostenida, centrado, a 3 cm del borde superior de la hoja.  El primer término aparece a dos interlíneas del título glosario, contra el margen izquierdo. Los términos se escriben con mayúscula sostenida seguidos de dos puntos y en orden alfabético. La definición correspondiente se coloca después de los dos puntos, se deja un espacio y se inicia con minúscula. Si ocupa más de un renglón, el segundo y los subsiguientes comienzan contra el margen izquierdo. Entre término y término se deja una interlínea. Su uso es opcional
\begin{itemize}

    \item \textbf{Decision Tree}: modelo basado en divisiones jerárquicas del espacio de características para realizar predicciones mediante reglas simples.
    \item \textbf{Estrato (Colombia)}: clasificación socioeconómica oficial de los inmuebles residenciales en rangos de 1 a 6, utilizada para focalización de subsidios y cobros de servicios públicos. En mercado inmobiliario funciona como un proxy de nivel socioeconómico de zona \cite{dane_estratificacion}.
    \item \textbf{Geohash}: sistema de codificación espacial que divide el territorio en celdas jerárquicas. Útil para agregación espacial, análisis geográfico y almacenamiento eficiente de coordenadas.
    \item \textbf{KNN (K-Nearest Neighbors)}: algoritmo basado en distancias que predice valores utilizando los promedios de los vecinos más cercanos.
    \item \textbf{Lasso}: regresión lineal con regularización L1 que produce modelos más simples mediante la penalización y posible eliminación de coeficientes.
    \item \textbf{LightGBM}: algoritmo de \textit{gradient boosting} basado en histogramas, optimizado para grandes volúmenes de datos y alta velocidad de entrenamiento \cite{bigdata2019realestate}.
    \item \textbf{Linear Regression}: modelo estadístico que estima una relación lineal entre variables independientes y la variable objetivo.
    \item \textbf{MAE (Mean Absolute Error)}: métrica que representa el error absoluto promedio entre valores reales y predicciones. Menos sensible a valores extremos que el RMSE.
    \item \textbf{Método hedónico}: técnica econométrica que estima el valor de un bien a partir de sus características intrínsecas y extrínsecas, ampliamente utilizada en valoración inmobiliaria.
    \item \textbf{One-Hot Encoding}: método de codificación categórica que convierte cada categoría en una columna binaria para su uso en modelos de ML.
    \item \textbf{OSM (OpenStreetMap)}: base de datos geográfica colaborativa abierta que almacena vías, edificios y puntos de interés, usada para enriquecer el contexto espacial en este estudio \cite{osm_poi}.
    \item \textbf{POI (Point of Interest)}: punto geográfico que representa lugares de interés como colegios, hospitales, parques o comercios. Utilizado para modelar accesibilidad y amenidades.
    \item \textbf{PostGIS}: extensión espacial para PostgreSQL que permite almacenar y procesar geometrías, realizar consultas espaciales y manejar proyecciones \cite{postgis_manual}.
    \item \textbf{Random Forest}: modelo de ensamble basado en múltiples árboles de decisión entrenados de manera independiente, robusto y resistente al sobreajuste \cite{kim2018machinelearning}.
    \item \textbf{Ridge}: regresión lineal con regularización L2, útil para mitigar problemas de multicolinealidad entre variables.
    \item \textbf{RMSE (Root Mean Squared Error)}: métrica que mide la raíz del error cuadrático medio. Penaliza de forma más fuerte los errores grandes y es la métrica principal utilizada en este estudio.
    \item \textbf{R\textsuperscript{2} (Coeficiente de determinación)}: indicador que mide la proporción de variabilidad explicada por el modelo. Valores cercanos a 1 indican alto poder explicativo.
    \item \textbf{SHAP (SHapley Additive exPlanations)}: método de interpretabilidad basado en teoría de juegos que asigna a cada característica su contribución al valor predicho.
    \item \textbf{SRID}: identificador numérico de un sistema de referencia espacial (p.\ ej., 4326 o 3857). Determina cómo interpretar y transformar geometrías \cite{postgis_manual}.
    \item \textbf{SVR (Support Vector Regression)}: variante de máquinas de soporte vectorial utilizada para tareas de regresión.
    \item \textbf{UPZ (Unidad de Planeamiento Zonal)}: división administrativa en Bogotá que agrupa sectores con características urbanísticas y socioeconómicas homogéneas.
    \item \textbf{Validación cruzada (Cross-validation)}: técnica que divide los datos en múltiples pliegues para entrenar y evaluar un modelo de manera repetida, mejorando la estabilidad de las métricas obtenidas.
    \item \textbf{Web scraping}: técnica de extracción automatizada de datos desde sitios web mediante scripts o herramientas especializadas.
    \item \textbf{Web Mercator (EPSG:3857)}: proyección cartográfica pseudo-mercator que representa coordenadas en metros, ampliamente usada en mapas web \cite{epsg3857}.
    \item \textbf{WGS84 (EPSG:4326)}: sistema geodésico mundial basado en latitud y longitud en grados. Es el sistema de referencia de GPS y de la mayoría de datasets geoespaciales \cite{epsg4326}.
    \item \textbf{XGBoost}: algoritmo de \textit{gradient boosting} basado en árboles de decisión. Destacado por su rendimiento y capacidad de generalización en múltiples contextos \cite{park2015housing}.

\end{itemize}
