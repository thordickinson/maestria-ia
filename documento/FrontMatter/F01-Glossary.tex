% Lista alfabética de términos y sus definiciones o explicaciones necesarios para la comprensión del documento. La existencia de un glosario no justifica la omisión de una explicación en el texto la primera vez que aparece un término. El título glosario se escribe en mayúscula sostenida, centrado, a 3 cm del borde superior de la hoja.  El primer término aparece a dos interlíneas del título glosario, contra el margen izquierdo. Los términos se escriben con mayúscula sostenida seguidos de dos puntos y en orden alfabético. La definición correspondiente se coloca después de los dos puntos, se deja un espacio y se inicia con minúscula. Si ocupa más de un renglón, el segundo y los subsiguientes comienzan contra el margen izquierdo. Entre término y término se deja una interlínea. Su uso es opcional


\begin{itemize}
    \item \textbf{Método hedónico}: Técnica econométrica que estima el valor de un bien a partir de sus características intrínsecas y extrínsecas. En el caso de los bienes raíces, este método permite evaluar cómo factores como el tamaño, ubicación y calidad afectan el precio de una propiedad.

    \item \textbf{Web scraping}: Técnica utilizada para extraer datos de sitios web de manera automatizada mediante el uso de herramientas y bibliotecas de programación, como BeautifulSoup o Scrapy en Python.

    \item \textbf{PSO (Particle Swarm Optimization)}: Algoritmo de optimización basado en la inteligencia colectiva de grupos, inspirado en el comportamiento de enjambres como aves o peces. Se utiliza para resolver problemas complejos mediante iteraciones en busca de soluciones óptimas.

    \item \textbf{Especulación}: Práctica económica que consiste en la compra de bienes, como propiedades inmobiliarias, con el objetivo de obtener ganancias a través del aumento de su precio, a menudo contribuyendo a la inflación de precios y dificultando el acceso a dichos bienes para sectores de bajos ingresos.
    
    \item \textbf{Estrato (Colombia)}: clasificación socioeconómica oficial de los inmuebles residenciales en rangos 1 a 6, usada para la focalización de subsidios y cobros de servicios públicos. En mercado inmobiliario sirve como proxy de nivel socioeconómico de zona. \cite{dane_estratificacion}

    \item \textbf{Método hedónico}: Técnica econométrica que estima el valor de un bien a partir de sus características intrínsecas y extrínsecas. En el caso de los bienes raíces, este método permite evaluar cómo factores como el tamaño, ubicación y calidad afectan el precio de una propiedad.

    \item \textbf{OSM (OpenStreetMap)}: base de datos geográfica colaborativa abierta que almacena vías, edificios y POIs, entre otros. Fuente empleada para enriquecer el contexto espacial (amenidades por radio). \cite{osm_poi}

    \item \textbf{POI (Point of Interest)}: punto de interés geográfico que representa lugares relevantes (p. ej., colegios, hospitales, parques, comercio). Usado para medir accesibilidad y amenidades cercanas a una propiedad. \cite{osm_poi}

    \item \textbf{PSO (Particle Swarm Optimization)}: Algoritmo de optimización basado en la inteligencia colectiva de grupos, inspirado en el comportamiento de enjambres como aves o peces. Se utiliza para resolver problemas complejos mediante iteraciones en busca de soluciones óptimas.

    \item \textbf{SRID}: identificador numérico de un sistema de referencia espacial (p. ej., 4326 o 3857). En PostGIS determina cómo interpretar y transformar geometrías entre sistemas. \emph{Ver Anexo~\ref{annex:crs}}. \cite{postgis_manual}

    \item \textbf{Web Mercator (EPSG:3857)}: proyección cartográfica pseudo-mercator usada por la mayoría de mapas web. Expresa coordenadas en metros, útil para cálculos de distancia en entornos urbanos. \emph{Ver Anexo~\ref{annex:crs}}. \cite{epsg3857}

    \item \textbf{Web scraping}: Técnica utilizada para extraer datos de sitios web de manera automatizada mediante el uso de herramientas y bibliotecas de programación, como BeautifulSoup o Scrapy en Python.

    \item \textbf{WGS84 (EPSG:4326)}: sistema geodésico mundial que define la forma de la Tierra y un sistema de coordenadas geográficas en grados (latitud/longitud). Base de GPS y de la mayoría de datasets geoespaciales. \emph{Ver Anexo~\ref{annex:crs}}. \cite{epsg4326}
\end{itemize}