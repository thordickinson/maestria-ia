Este trabajo presenta un modelo de estimación de precios de apartamentos en Bogotá utilizando técnicas de aprendizaje automático y datos abiertos enriquecidos con información geoespacial. El objetivo es evaluar en qué medida las variables de entorno urbano —como proximidad a servicios y equipamientos— aportan capacidad explicativa adicional frente a un modelo basado únicamente en características estructurales del inmueble.

Se integraron fuentes de datos provenientes de plataformas públicas, capas geográficas oficiales y puntos de interés de OpenStreetMap. Tras un proceso de limpieza, imputación y transformación, se consolidó un conjunto de más de 27{,}000 registros con variables estructurales, socioespaciales y de localización. Se exploró su comportamiento mediante análisis estadístico y visual, incluyendo transformaciones logarítmicas y mapas de distribución espacial.

Se entrenaron y compararon diversos modelos predictivos (Regresión Lineal, Ridge, Lasso, Random Forest, LightGBM y XGBoost), evaluados mediante validación cruzada de diez pliegues. El modelo base, construido únicamente con variables estructurales, alcanzó un RMSE promedio de 0.1424 y un $R^{2} = 0.9467$, superando estadísticamente los modelos enriquecidos. La ubicación y el área del inmueble emergieron como los principales determinantes del precio, con latitud, longitud y UPZ representando de forma implícita patrones socioeconómicos y territoriales de la ciudad.

El modelo enriquecido con variables geoespaciales obtuvo un RMSE de 0.1494 y un $R^{2} = 0.9387$, mientras que el modelo enriquecido optimizado —con un subconjunto reducido de variables seleccionadas mediante SHAP— alcanzó un RMSE de 0.1480. Aunque estas variables no mejoraron la precisión, sí aportaron mayor interpretabilidad del contexto urbano y confirmaron la relevancia estructural de la localización en la formación del precio.

Como complemento aplicado, se implementó una aplicación web basada en FastAPI y ReactJS, que permite a los usuarios ingresar las características de un apartamento, obtener una estimación del precio y visualizar información estadística contextual de su entorno urbano.

\textbf{Palabras clave:} apartamentos, precios inmobiliarios, aprendizaje automático, XGBoost, datos abiertos, geolocalización, Bogotá.

\subsection*{Abstract}

This work presents an apartment price estimation model for Bogotá using machine learning techniques and open data enriched with geospatial information. The objective is to assess the extent to which urban-environment variables—such as proximity to services and amenities—provide additional explanatory power compared to a model based solely on structural characteristics of the property.

Data sources from public platforms, official geographic layers, and OpenStreetMap points of interest were integrated. After a process of cleaning, imputation, and transformation, a dataset of more than 27{,}000 records was consolidated, incorporating structural, socio-spatial, and location-based variables. Its behavior was explored through statistical and visual analysis, including logarithmic transformations and spatial distribution maps.

Multiple predictive models (Linear Regression, Ridge, Lasso, Random Forest, LightGBM, and XGBoost) were trained and compared using ten-fold cross-validation. The baseline model, built using only structural variables, achieved a mean RMSE of 0.1424 and $R^{2} = 0.9467$, statistically outperforming the enriched models. Location and floor area emerged as the main determinants of price, with latitude, longitude, and UPZ implicitly capturing socioeconomic and territorial patterns of the city.

The model enriched with geospatial variables obtained an RMSE of 0.1494 and $R^{2} = 0.9387$, while the optimized enriched model—using a reduced subset of variables selected through SHAP—achieved an RMSE of 0.1480. Although these variables did not improve predictive accuracy, they enhanced interpretability of the urban context and confirmed the structural relevance of location in price formation.

As an applied component, a web application was implemented using FastAPI and ReactJS, enabling users to input apartment characteristics, obtain a price estimate, and visualize contextual statistical information about the surrounding urban environment.

\textbf{Keywords:} apartments, housing prices, machine learning, XGBoost, open data, geolocation, Bogotá.
