\section*{Resumen}

El presente trabajo desarrolla un modelo de estimación de precios de \textbf{apartamentos} en la ciudad de Bogotá utilizando técnicas de aprendizaje automático y datos abiertos enriquecidos con información geoespacial. El estudio parte del reconocimiento de que el déficit habitacional y la desigualdad en el acceso a vivienda digna constituyen problemáticas estructurales del país, y que la falta de herramientas analíticas transparentes limita la comprensión del mercado inmobiliario.

Se integraron fuentes de datos provenientes de plataformas públicas, capas geográficas oficiales y puntos de interés de OpenStreetMap. Tras un proceso de limpieza, imputación y normalización, se construyó un conjunto de datos de más de 27.000 registros, que permitió entrenar y comparar diferentes modelos predictivos, incluyendo regresión lineal, Ridge, Lasso, SVR, Random Forest, LightGBM y XGBoost. La evaluación se realizó mediante validación cruzada de diez pliegues, utilizando como métrica principal la raíz del error cuadrático medio (RMSE).

El modelo base, entrenado únicamente con variables estructurales, alcanzó un RMSE promedio de 0.151 y un $R^2$ de 0.965, evidenciando alta estabilidad y capacidad explicativa. El modelo enriquecido con variables geoespaciales obtuvo un RMSE de 0.165 y un $R^2$ de 0.958, mostrando un rendimiento comparable pero con mayor interpretabilidad espacial. 

Finalmente, se desarrolló una \textbf{aplicación web} compuesta por un backend en \textit{FastAPI} y un frontend en \textit{ReactJS}, que permite al usuario ingresar las características de un apartamento, estimar su precio y consultar información estadística contextual sobre su entorno urbano. 

\textbf{Palabras clave:} apartamentos, precios inmobiliarios, aprendizaje automático, XGBoost, datos abiertos, geolocalización, Bogotá.
