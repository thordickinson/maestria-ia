Este trabajo presenta un modelo de estimación de precios de \textbf{apartamentos} en Bogotá utilizando técnicas de aprendizaje automático y datos abiertos enriquecidos con información geoespacial. El objetivo es evaluar en qué medida las variables de entorno urbano —como proximidad a servicios y equipamientos— aportan capacidad explicativa adicional frente a un modelo basado únicamente en características estructurales del inmueble.

Se integraron fuentes de datos provenientes de plataformas públicas, capas geográficas oficiales y puntos de interés de OpenStreetMap. Tras un proceso de limpieza, imputación y transformación, se consolidó un conjunto de más de 27.000 registros con variables estructurales, socioespaciales y de localización. Se exploró su comportamiento mediante análisis estadístico y visual, incluyendo transformaciones logarítmicas y mapas de distribución espacial.

Se entrenaron y compararon diversos modelos predictivos (Regresión Lineal, Ridge, Lasso, SVR, Random Forest, LightGBM y XGBoost), evaluados mediante validación cruzada de diez pliegues. El modelo base, construido únicamente con variables estructurales, alcanzó un \textbf{RMSE promedio de 0.1514} y un \textbf{$R^2 = 0.965$}, superando estadísticamente los modelos enriquecidos. La ubicación y el área del inmueble emergieron como los principales determinantes del precio, con latitud, longitud y UPZ representando de forma implícita patrones socioeconómicos y territoriales de la ciudad.

El modelo enriquecido con variables geoespaciales obtuvo un \textbf{RMSE de 0.1705} y un \textbf{$R^2 = 0.956$}, mientras que el modelo enriquecido optimizado —con un subconjunto reducido de variables seleccionadas mediante SHAP— alcanzó un \textbf{RMSE de 0.1584}. Aunque estas variables no mejoraron la precisión, sí aportaron mayor interpretabilidad del contexto urbano y confirmaron la relevancia estructural de la localización en la formación del precio.

Como complemento aplicado, se implementó una \textbf{aplicación web} basada en \textit{FastAPI} y \textit{ReactJS}, que permite a los usuarios ingresar las características de un apartamento, obtener una estimación del precio y visualizar información estadística contextual de su entorno urbano.

\textbf{Palabras clave:} apartamentos, precios inmobiliarios, aprendizaje automático, XGBoost, datos abiertos, geolocalización, Bogotá.


\subsection*{Abstract}

This study develops a machine learning model to estimate apartment prices in Bogotá using open data and geospatial information. A dataset of more than 27,000 records was constructed by integrating structural property attributes, administrative divisions, and points of interest from OpenStreetMap. Several regression models were evaluated using ten-fold cross-validation, with XGBoost achieving the best performance. The baseline model, trained only on structural features, obtained a mean RMSE of 0.1514 and $R^2 = 0.965$, outperforming geospatially enriched variants. Results indicate that apartment prices are primarily determined by two factors: floor area and location. Geospatial variables did not improve accuracy significantly, as their information is largely encoded in the property’s geographic coordinates and administrative units such as UPZ. However, SHAP analysis showed that these variables enhance interpretability by revealing the spatial structure of Bogotá’s housing market. A web application was also implemented using FastAPI and ReactJS to deliver model predictions and contextual urban information to end-users.

\textbf{Keywords:} housing prices, machine learning, open data, geospatial analysis, XGBoost, Bogotá.
