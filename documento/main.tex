%        1         2         3         4         5         6         7         8         9         0         1         2
%23456789012345678901234567890123456789012345678901234567890123456789012345678901234567890123456789012345678901234567890
\documentclass[12pt]{MIA-USA}

\usepackage{layout}
\usepackage{subcaption}
\usepackage{paralist}
\usepackage{breakcites}
\usepackage{pgfgantt}
\usepackage{pdflscape}

%% Semana del 16 de junio ¡ANTEPROYECTO! 12 clases
% Cada trabajo vale lo mismo que el anterior. 16%
% 

% Trabajos, aproximadamente cada dos semanas.
% Problema y justificación [Título] (13 de marzo)
% Objetivos [General y espécificos (No más de 4)] (3 de abril)
%   * Cada objetivo espécifico se debe soportar por un capítulo en el proyecto.
% Estado del arte (17 de abril)
% Metodología (8 de mayo)
% Cronograma y presupuesto (22 de mayo)
% Resultados esperados (5 de junio)


% La introducción y los resumenes es lo último que se hace. ()

%Para las referencias usar mendeley o google académico
% * Usar artículos recientes en lo posible
% * sci-hub.se? (Ver los artículos copiar el DOI)


\usepackage[pagebackref=true,breaklinks=true]{hyperref}
\hypersetup{
	pdftitle={Documento de trabajo Makers}, 
	pdfauthor={MAKERS}, 
	pdfsubject={}, 
	pdfcreator={Dickinson Arismendy}, 
	colorlinks=true,
	linkcolor=GrisUno,
	citecolor=AzulInstitucional,
	filecolor=AzulClaro,
	urlcolor=AzulInstitucional,
	linktoc=all
}

\graphicspath{{Imagenes/}}



\title{Predicción de Precios de Viviendas en Bogotá Usando Machine Learning y Datos Enriquecidos}
\author{Dickinson Román Arismendy Torres}
\documenttype{Maestr\'ia en Inteligencia Artificial}

\advisor{Phd. Juan Pablo Ospina Lopez}

% Si hay un coasesor 
% \coadvisor{Nombres y apellidos Co-Asesor}


% Aqui comienza el documento
% - -- --- ----- ------- ------- ----- --- -- -
\begin{document}
    % \layout{}
	
    \maketitle
    
    % Resumen
    \chapter*{Resumen}
    \section*{Resumen}

El presente trabajo desarrolla un modelo de estimación de precios de \textbf{apartamentos} en la ciudad de Bogotá utilizando técnicas de aprendizaje automático y datos abiertos enriquecidos con información geoespacial. El estudio parte del reconocimiento de que el déficit habitacional y la desigualdad en el acceso a vivienda digna constituyen problemáticas estructurales del país, y que la falta de herramientas analíticas transparentes limita la comprensión del mercado inmobiliario.

Se integraron fuentes de datos provenientes de plataformas públicas, capas geográficas oficiales y puntos de interés de OpenStreetMap. Tras un proceso de limpieza, imputación y normalización, se construyó un conjunto de datos de más de 27.000 registros, que permitió entrenar y comparar diferentes modelos predictivos, incluyendo regresión lineal, Ridge, Lasso, SVR, Random Forest, LightGBM y XGBoost. La evaluación se realizó mediante validación cruzada de diez pliegues, utilizando como métrica principal la raíz del error cuadrático medio (RMSE).

El modelo base, entrenado únicamente con variables estructurales, alcanzó un RMSE promedio de 0.151 y un $R^2$ de 0.965, evidenciando alta estabilidad y capacidad explicativa. El modelo enriquecido con variables geoespaciales obtuvo un RMSE de 0.165 y un $R^2$ de 0.958, mostrando un rendimiento comparable pero con mayor interpretabilidad espacial. 

Finalmente, se desarrolló una \textbf{aplicación web} compuesta por un backend en \textit{FastAPI} y un frontend en \textit{ReactJS}, que permite al usuario ingresar las características de un apartamento, estimar su precio y consultar información estadística contextual sobre su entorno urbano. 

\textbf{Palabras clave:} apartamentos, precios inmobiliarios, aprendizaje automático, XGBoost, datos abiertos, geolocalización, Bogotá.

    % Crea la tabla de contenidos a partir de la estructura del documento
    \tableofcontents
    % Crea la lista de figuras
    \listoffigures
    % Crea la lista de tablas
    \listoftables
    
    % A partir de este punto se crea la estructura del documento para el espacio de práctica con la siguiente 
    % jerarquía
    % chapter (Capitulo)
    %   - section (sección)
    %       - subsection (subsección)
    %           - subsubsection (subsubsección)
    
    
    % Glosario
    \chapter*{Glosario}
    % Lista alfabética de términos y sus definiciones o explicaciones necesarios para la comprensión del documento. La existencia de un glosario no justifica la omisión de una explicación en el texto la primera vez que aparece un término. El título glosario se escribe en mayúscula sostenida, centrado, a 3 cm del borde superior de la hoja.  El primer término aparece a dos interlíneas del título glosario, contra el margen izquierdo. Los términos se escriben con mayúscula sostenida seguidos de dos puntos y en orden alfabético. La definición correspondiente se coloca después de los dos puntos, se deja un espacio y se inicia con minúscula. Si ocupa más de un renglón, el segundo y los subsiguientes comienzan contra el margen izquierdo. Entre término y término se deja una interlínea. Su uso es opcional


\begin{itemize}
    \item \textbf{Método hedónico}: Técnica econométrica que estima el valor de un bien a partir de sus características intrínsecas y extrínsecas. En el caso de los bienes raíces, este método permite evaluar cómo factores como el tamaño, ubicación y calidad afectan el precio de una propiedad.

    \item \textbf{Web scraping}: Técnica utilizada para extraer datos de sitios web de manera automatizada mediante el uso de herramientas y bibliotecas de programación, como BeautifulSoup o Scrapy en Python.

    \item \textbf{PSO (Particle Swarm Optimization)}: Algoritmo de optimización basado en la inteligencia colectiva de grupos, inspirado en el comportamiento de enjambres como aves o peces. Se utiliza para resolver problemas complejos mediante iteraciones en busca de soluciones óptimas.

    \item \textbf{Especulación}: Práctica económica que consiste en la compra de bienes, como propiedades inmobiliarias, con el objetivo de obtener ganancias a través del aumento de su precio, a menudo contribuyendo a la inflación de precios y dificultando el acceso a dichos bienes para sectores de bajos ingresos.
    
    \item \textbf{Estrato (Colombia)}: clasificación socioeconómica oficial de los inmuebles residenciales en rangos 1 a 6, usada para la focalización de subsidios y cobros de servicios públicos. En mercado inmobiliario sirve como proxy de nivel socioeconómico de zona. \cite{dane_estratificacion}

    \item \textbf{Método hedónico}: Técnica econométrica que estima el valor de un bien a partir de sus características intrínsecas y extrínsecas. En el caso de los bienes raíces, este método permite evaluar cómo factores como el tamaño, ubicación y calidad afectan el precio de una propiedad.

    \item \textbf{OSM (OpenStreetMap)}: base de datos geográfica colaborativa abierta que almacena vías, edificios y POIs, entre otros. Fuente empleada para enriquecer el contexto espacial (amenidades por radio). \cite{osm_poi}

    \item \textbf{POI (Point of Interest)}: punto de interés geográfico que representa lugares relevantes (p. ej., colegios, hospitales, parques, comercio). Usado para medir accesibilidad y amenidades cercanas a una propiedad. \cite{osm_poi}

    \item \textbf{PSO (Particle Swarm Optimization)}: Algoritmo de optimización basado en la inteligencia colectiva de grupos, inspirado en el comportamiento de enjambres como aves o peces. Se utiliza para resolver problemas complejos mediante iteraciones en busca de soluciones óptimas.

    \item \textbf{SRID}: identificador numérico de un sistema de referencia espacial (p. ej., 4326 o 3857). En PostGIS determina cómo interpretar y transformar geometrías entre sistemas. \emph{Ver Anexo~\ref{annex:crs}}. \cite{postgis_manual}

    \item \textbf{Web Mercator (EPSG:3857)}: proyección cartográfica pseudo-mercator usada por la mayoría de mapas web. Expresa coordenadas en metros, útil para cálculos de distancia en entornos urbanos. \emph{Ver Anexo~\ref{annex:crs}}. \cite{epsg3857}

    \item \textbf{Web scraping}: Técnica utilizada para extraer datos de sitios web de manera automatizada mediante el uso de herramientas y bibliotecas de programación, como BeautifulSoup o Scrapy en Python.

    \item \textbf{WGS84 (EPSG:4326)}: sistema geodésico mundial que define la forma de la Tierra y un sistema de coordenadas geográficas en grados (latitud/longitud). Base de GPS y de la mayoría de datasets geoespaciales. \emph{Ver Anexo~\ref{annex:crs}}. \cite{epsg4326}
\end{itemize}

    % Introduction
    \chapter{Introducci\'on}
    % En ella, el autor presenta y señala la importancia, el origen (los antecedentes teóricos y prácticos), los objetivos, los alcances, las limitaciones, la metodología empleada, el significado que el estudio tiene en el avance del campo respectivo y su aplicación en el área investigada.

% Entre 60 y 80 páginas

La vivienda constituye uno de los pilares fundamentales del bienestar social y del desarrollo urbano. Su acceso, calidad y localización inciden directamente en la equidad, la movilidad y la planificación del territorio. En Colombia, el déficit habitacional, tanto en su dimensión cuantitativa como cualitativa, representa uno de los principales retos sociales y económicos del país. Según Castillo (2004), el déficit no solo se relaciona con la cantidad insuficiente de viviendas, sino también con la precariedad en las condiciones del entorno y los servicios urbanos \cite{castillo2004anotaciones}. En ciudades como Bogotá, el crecimiento desordenado y la urbanización informal han intensificado la segregación socioespacial y la exclusión \cite{castillo2004anotaciones}.

La problemática se agrava por la falta de acceso a crédito hipotecario y la especulación inmobiliaria, factores que distorsionan el mercado y encarecen los precios, especialmente para los hogares de bajos ingresos. Se estima que más del 80\% de estos hogares no tiene acceso a financiamiento formal, lo cual acentúa la desigualdad en el acceso a vivienda \cite{castillo2004anotaciones}. A su vez, políticas de subsidio como \textit{Mi Casa Ya} han experimentado una implementación irregular, afectando tanto a desarrolladores como a compradores y generando incertidumbre en el sector \cite{coyunturaVivienda2023}.

En este contexto, la vivienda también ha sido utilizada como un instrumento de especulación financiera y política, lo que dificulta la construcción de indicadores confiables sobre el valor real de los inmuebles. Los métodos tradicionales de valoración —basados en el juicio experto o en comparaciones de propiedades cercanas— tienden a reproducir sesgos de subjetividad y a depender de información incompleta. En la práctica, los agentes inmobiliarios suelen estimar el precio de una propiedad utilizando métricas simplificadas como el \textit{precio por metro cuadrado}, ajustando ligeramente por número de baños, estado o ubicación. Este enfoque refleja la estructura empírica del mercado, pero carece de una integración sistemática de factores espaciales, económicos y sociales.

A nivel internacional, la literatura reciente ha demostrado que el uso de técnicas de aprendizaje automático permite capturar relaciones no lineales y multidimensionales en la formación del precio de la vivienda. Estudios como \textit{yu2016realestate} \cite{dabreo2021realestate}, \textit{Big Data Analytics Predicting Real Estate Prices} \cite{bigdata2019realestate} y \textit{Machine and Deep Learning for Hedonic Real Estate Price Prediction} \cite{kim2018machinelearning} muestran que modelos como XGBoost, LightGBM y redes neuronales profundas pueden superar los enfoques hedónicos tradicionales, integrando grandes volúmenes de datos estructurales y contextuales con altos niveles de precisión. Sin embargo, la aplicación de estos enfoques en contextos latinoamericanos, y en particular en Colombia, sigue siendo limitada debido a la fragmentación de las fuentes de datos, la escasa disponibilidad de información espacial normalizada y la falta de estudios reproducibles.

En Colombia, los modelos automatizados de estimación de precios de vivienda se basan principalmente en información proveniente de portales inmobiliarios. Estos portales, además, presentan sesgos estructurales derivados de su modelo de negocio: las inmobiliarias y agentes deben pagar por paquetes de publicaciones, lo que incentiva la visibilización de los inmuebles de mayor valor, capaces de generar comisiones más altas. En consecuencia, los datos publicados reflejan de forma parcial la realidad del mercado, concentrándose en zonas de alto valor y dejando subrepresentados los sectores populares o periféricos.

El presente trabajo busca abordar esta brecha mediante el desarrollo de un modelo de estimación de precios de vivienda en Bogotá utilizando técnicas de aprendizaje automático y datos abiertos enriquecidos con información geoespacial. A diferencia de los enfoques convencionales, se propone integrar características estructurales del inmueble con variables de contexto derivadas de capas oficiales (barrios, UPZ, localidades, estratos y avalúos) y puntos de interés (hospitales, colegios, parques, comercio, entre otros), obtenidos desde OpenStreetMap y datos abiertos distritales.

La investigación se centra en tres aspectos complementarios: (i) la construcción de un conjunto de datos reproducible que combine información estructural y espacial, (ii) la comparación de modelos predictivos con y sin variables geográficas para evaluar su impacto en el rendimiento y la interpretabilidad, y (iii) el análisis de la importancia de las variables y la distribución del error para comprender los factores que determinan el precio de la vivienda. 

El documento se organiza de la siguiente manera: la \textbf{Sección de Metodología} describe las etapas de adquisición, limpieza, enriquecimiento y modelado de los datos; la \textbf{Sección de Resultados} presenta las métricas obtenidas en los modelos base y enriquecidos, junto con los análisis de importancia y error; la \textbf{Discusión} interpreta los resultados en el contexto del mercado inmobiliario bogotano y sus sesgos estructurales; y finalmente, la \textbf{Conclusión} sintetiza los hallazgos principales y propone líneas de trabajo futuro.



    % Problema de investigacion
    \chapter{Problema de investigaci\'on}
    El déficit habitacional en Colombia, caracterizado por la falta de acceso a vivienda digna y equitativa, es una problemática compleja que afecta a millones de personas. Según Castillo (2004), el déficit habitacional en el país incluye tanto aspectos cuantitativos, como la insuficiencia de viviendas disponibles, como cualitativos, relacionados con la calidad de las viviendas y su entorno \cite{castillo2004anotaciones}. Este problema es especialmente agudo en Bogotá, donde el crecimiento desorganizado y la urbanización informal han exacerbado la marginalidad y la exclusión \cite{castillo2004anotaciones}.

Adicionalmente, el acceso limitado al financiamiento hipotecario y la especulación inmobiliaria contribuyen a distorsionar el mercado, encareciendo los precios y excluyendo a los hogares de bajos ingresos. Más del 80\% de los hogares de bajos ingresos no tiene acceso a créditos hipotecarios, lo que agrava la inequidad en el acceso a la vivienda \cite{castillo2004anotaciones}. Este panorama se ve reflejado en los cambios recientes en las políticas de subsidios, como \textit{Mi Casa Ya}, cuya implementación irregular ha afectado tanto a desarrolladores como a compradores, aumentando la incertidumbre en el sector \cite{coyunturaVivienda2023}.

En paralelo, la vivienda también ha sido utilizada históricamente como un instrumento de especulación financiera y política. La falta de integración de datos cualitativos, como la calidad de los servicios públicos, la accesibilidad y el equipamiento urbano, limita la capacidad de los modelos actuales de predicción de precios para reflejar las dinámicas reales del mercado. Los modelos tradicionales, basados únicamente en características extraídas de portales inmobiliarios, no consideran los factores sociales y económicos que afectan directamente los precios, como destacó Walter (2023) en su análisis de la desaceleración del sector vivienda en Colombia \cite{coyunturaVivienda2023, fortaleciendoVivienda2024}.


    % Objetivos
    \chapter{Objetivos}
    % Incluye los objetivos general y específicos del proyecto. Debe procurar seguir una metodología de formulación de objetivos, por ejemplo S.M.A.R.T.

% Debe tener solo un verbo en infinitivo e incluír como lo van a medir a uno


\section{Objetivo general}

Desarrollar un modelo de estimación de precios de vivienda en Bogotá utilizando técnicas de aprendizaje automático y datos abiertos, que combine variables estructurales y geoespaciales para ofrecer predicciones confiables y explicativas. Además, implementar una herramienta web que permita a los usuarios consultar estimaciones personalizadas y visualizar estadísticas informativas sobre el contexto urbano del inmueble.

\section{Objetivos específicos}
\begin{itemize}
    \item Consolidar un conjunto de datos de viviendas en Bogotá a partir de fuentes públicas y bases de datos disponibles en línea, integrando información estructural (área, número de habitaciones, baños, parqueaderos, estrato) y espacial (ubicación geográfica, barrio, UPZ, proximidad a puntos de interés).
    \item Estandarizar, limpiar e imputar los datos garantizando su coherencia y calidad para el entrenamiento de modelos predictivos.
    \item Desarrollar y comparar distintos algoritmos de aprendizaje automático para la estimación de precios, seleccionando el modelo con mejor desempeño y capacidad explicativa.
    \item Incorporar variables geoespaciales y evaluar su impacto sobre la precisión y la interpretabilidad del modelo.
    \item Diseñar e implementar una utilidad web que permita al usuario ingresar las características de un inmueble y obtener una estimación de precio junto con información estadística contextual (por ejemplo, precios promedio en el sector, accesibilidad y servicios cercanos).
\end{itemize}

    
    % Trabajos previos
    \chapter{Estado del Arte}
    El análisis y la estimación del valor de los bienes raíces se ha consolidado en la última década como un componente estratégico en la gestión del mercado inmobiliario. La expansión de los portales digitales, el acceso a fuentes de información pública y el auge de la analítica de datos han impulsado el desarrollo de modelos que buscan determinar precios de vivienda de forma transparente, replicable y sustentada en evidencia cuantitativa, en contraste con los métodos tradicionales basados en juicios subjetivos o comparaciones locales. La revisión de la literatura especializada permite identificar cuatro grandes líneas de investigación, diferenciadas por su enfoque metodológico y su contribución al negocio inmobiliario.

Los primeros estudios aplicaron métodos estadísticos convencionales, como la regresión lineal múltiple o los modelos hedónicos, para explicar cómo las características físicas —área, número de habitaciones, baños, garajes o antigüedad— influyen en el valor de una vivienda. Investigaciones como las de \cite{yu2016realestate} y \cite{kim2018machinelearning} evidencian la utilidad de estos modelos en contextos homogéneos, donde la estructura del inmueble explica buena parte de la variabilidad del precio. No obstante, su capacidad predictiva disminuye en mercados urbanos complejos, donde factores como la ubicación o el entorno socioeconómico tienen un peso determinante.

Con el propósito de anticipar fluctuaciones de precios y evaluar impactos derivados de la coyuntura económica, autores como \cite{li2017realestate} y \cite{zhang2018realestate} incorporaron indicadores macroeconómicos —PIB, inflación, tasas hipotecarias, inversión en construcción o empleo— dentro de modelos predictivos. En estos trabajos se emplean técnicas de Support Vector Regression (SVR) y variantes optimizadas mediante Particle Swarm Optimization (PSO), orientadas a capturar relaciones no lineales y generar proyecciones útiles para la planeación estratégica del sector inmobiliario.

El crecimiento de los portales inmobiliarios y la disponibilidad de datos estructurados a gran escala promovieron la adopción de técnicas más avanzadas de aprendizaje automático. \cite{park2015housing} introducen algoritmos de clasificación como C4.5, RIPPER, Naïve Bayes y AdaBoost para analizar datos de múltiples fuentes (MLS, tasas hipotecarias, calidad educativa), destacando su potencial para apoyar la toma de decisiones de agentes y compradores. De manera complementaria, \cite{bigdata2019realestate} exploraron el uso de Random Forest, LASSO y Gradient Boosting sobre conjuntos de datos con decenas de variables explicativas, evidenciando el papel del aprendizaje automático en la automatización de la valoración masiva de propiedades y en la reducción de la subjetividad en las tasaciones.

La integración de arquitecturas neuronales profundas y técnicas de reducción de dimensionalidad ha potenciado la capacidad de los modelos para representar patrones complejos entre variables estructurales, económicas y espaciales. \cite{mostofi2022realestate} proponen un enfoque híbrido PCA–DNN para mejorar la predicción de precios en mercados heterogéneos, mientras que \cite{dabreo2021realestate} aplican redes neuronales y métodos de vecinos más cercanos para estimar precios urbanos en tiempo real. Estas aproximaciones marcan una transición desde modelos descriptivos hacia sistemas predictivos aplicables a portales inmobiliarios, procesos de tasación automatizada y gestión de carteras de inversión.

En el contexto colombiano, \cite{PerezRave2020ApartmentPricesColombia} modelan el precio de apartamentos ofertados en línea para Medellín, Envigado y Sabaneta usando más de 15\,000 anuncios con variables estructurales, de entorno y atributos derivados de texto. Comparan regresión lineal, árboles de decisión, Random Forest y esquemas de \textit{bagging}, encontrando desempeños cercanos a $R^{2} \approx 0{,}99$ y mostrando que área construida, número de baños, estrato y el precio medio por metro cuadrado de la zona concentran la mayor parte del poder explicativo. De forma similar, la tesis de \cite{MedinaGiraldo2023PrediccionPreciosMedellin} construye un modelo de valoración automática para viviendas en Medellín y el Área Metropolitana a partir de anuncios del portal Properati, comparando regresión lineal, árboles de decisión, Random Forest, $k$-NN y XGBoost; el mejor desempeño se obtiene con Random Forest, con $R^{2}\approx 0{,}81$ y un MAPE cercano al 14\,\%, proponiéndose como herramienta operativa para usuarios finales, portales e instituciones fiscales.

A nivel latinoamericano, \cite{Tapia2025AVMChileLightGBM} comparan un modelo hedónico espacial SAR con un AVM basado en LightGBM para la Región Metropolitana de Santiago de Chile, incorporando atributos estructurales, variables de entorno y \textit{features} visuales extraídos mediante CNN. El estudio concluye que el modelo de aprendizaje automático con variables de imagen mejora de forma notable la precisión frente al modelo hedónico, y mediante técnicas de interpretabilidad evidencia interacciones no lineales entre atributos físicos, calidad de colegios, accesibilidad y características del barrio, reforzando la relevancia de integrar información espacial y contextual en los sistemas de valoración.

La literatura revisada muestra una evolución desde modelos explicativos de carácter estadístico hacia modelos predictivos orientados a la gestión y la toma de decisiones empresariales en el mercado inmobiliario. Los avances recientes apuntan a combinar precisión, interpretabilidad y escalabilidad, y ponen de relieve la importancia de considerar variables espaciales y de entorno en mercados urbanos latinoamericanos. En particular, los trabajos desarrollados en Colombia y Chile \cite{PerezRave2020ApartmentPricesColombia,MedinaGiraldo2023PrediccionPreciosMedellin,Tapia2025AVMChileLightGBM} evidencian el potencial de los modelos de aprendizaje automático para capturar la heterogeneidad espacial y soportar aplicaciones prácticas de tasación masiva. En la Tabla~\ref{tab:state_of_art} se resumen algunos de los trabajos más representativos, destacando sus contextos de aplicación, los modelos utilizados y las métricas de evaluación reportadas.

\begin{table}[H]
\centering
\caption{Estudios representativos sobre predicción de precios inmobiliarios}
\label{tab:state_of_art}
\begin{tabular}{p{2cm} p{6cm} p{3.2cm} p{2.1cm}}
\hline
\textbf{Estudio} & \textbf{Contexto} & \textbf{Modelos} & \textbf{Métricas} \\ \hline
\cite{li2017realestate} & Mercado chino; predicción de variaciones de precios basada en indicadores macroeconómicos. & Regresión múltiple, ANN, SVM & RMSE, $R^{2}$ \\
\cite{zhang2018realestate} & Predicción de precios mediante optimización de hiperparámetros con PSO. & PSO–SVM, SVM, BP & MAPE, error relativo \\
\cite{kim2018machinelearning} & Comparación entre modelos hedónicos, de ML y \textit{deep learning}. & Hedonic regression, ANN, CNN & RMSE, MAE, $R^{2}$ \\
\cite{bigdata2019realestate} & Ames (Iowa); análisis de datos masivos y selección de variables relevantes. & Linear Regression, LASSO, Random Forest, Gradient Boosting & RMSE, MAE, $R^{2}$ \\
\cite{wang2019svr} & Predicción del precio inmobiliario mediante SVR. & SVR, BPNN & MAE, MAPE, RMSE \\
\cite{yu2016realestate} & Modelado de precios residenciales usando regresión y clasificación. & Linear Regression, Decision Tree, Random Forest, SVM & RMSE, MAE, Accuracy \\
\cite{dabreo2021realestate} & Aplicación práctica de ML para precios en entornos urbanos. & Linear Regression, KNN, Random Forest, SVM & RMSE, MAE \\
\cite{mostofi2022realestate} & Mercado turco; predicción con \textit{deep learning} y reducción de dimensionalidad. & DNN, SRA-DNN, PCA-DNN & MSE, MAE, MAPE \\
\cite{PerezRave2020ApartmentPricesColombia} & Medellín, Envigado y Sabaneta (Colombia); anuncios de apartamentos en portales web. & Linear Regression, árboles, Random Forest, Bagging & $R^{2}$, RMSE \\
\cite{MedinaGiraldo2023PrediccionPreciosMedellin} & Medellín y Área Metropolitana; datos de Properati. & Linear Regression, Decision Tree, Random Forest, KNN, XGBoost & $R^{2}$, MAPE \\
\cite{Tapia2025AVMChileLightGBM} & Región Metropolitana de Santiago (Chile); AVM con variables espaciales y de imagen. & LightGBM, modelo hedónico SAR & RMSE, MAE, $R^{2}$ \\
\hline
\end{tabular}
\end{table}

\noindent
La revisión presentada permite establecer que la investigación internacional ha logrado avances considerables en precisión y capacidad predictiva, aunque persisten desafíos en términos de acceso a datos, reproducibilidad y adaptabilidad a contextos locales. A partir de estos antecedentes, la presente investigación se enfoca en desarrollar un modelo predictivo aplicable al mercado de apartamentos en Bogotá, utilizando datos abiertos y variables espaciales derivadas de fuentes como OpenStreetMap y el portal de Datos Abiertos de la Alcaldía de Bogotá, lo cual se detalla en la siguiente sección metodológica.

    
    % Justificación
    % \chapter{Justificaci\'on}
    % % describir brevemente que just¿ifica el desarrollo del proyecto de investigación

% incluir el alcance que tendrá el proyecto a desarrollar

La vivienda es un derecho fundamental reconocido en la Constitución de Colombia \cite{ConstitucionPoliticaColombia1991}, pero la inequidad en el acceso, el déficit habitacional y la especulación inmobiliaria muestran que este derecho no se garantiza de manera efectiva. Desarrollar un modelo predictivo robusto, que incorpore variables estructuradas y contextuales, podría ayudar a mitigar los problemas asociados al mercado inmobiliario. Este enfoque no solo permitiría mejorar las estimaciones de precios, sino también diseñar políticas públicas más efectivas y basadas en evidencia \cite{fortaleciendoVivienda2024, castillo2004anotaciones, coyunturaVivienda2023}.

La predicción precisa de precios de viviendas es un aspecto crucial en el mercado inmobiliario, no solo para compradores y vendedores, sino también para inversionistas, desarrolladores y gobiernos locales. Cuando las estimaciones de precios son imprecisas o están basadas en datos incompletos, se generan varios problemas que afectan tanto al mercado como a los actores involucrados.

\subsection*{Decisiones de compra y venta desinformadas}
Los propietarios que subestiman el valor de su vivienda pueden venderla por debajo de su verdadero precio de mercado, perdiendo oportunidades de obtener una mayor ganancia. Por otro lado, los compradores que reciben estimaciones infladas pueden adquirir propiedades por encima de su valor real, enfrentando dificultades para revenderlas o recuperar la inversión. Este fenómeno afecta directamente la equidad y la transparencia del mercado, creando desconfianza entre los participantes.

\subsection*{Desigualdad en la accesibilidad a la vivienda}
Cuando las estimaciones de precios no son precisas, las zonas de alto crecimiento pueden ser sobrevaloradas, y las de menor crecimiento pueden ser infravaloradas, lo que provoca una distorsión en la accesibilidad a la vivienda. Esto puede llevar a que sectores de la población, especialmente los de menores ingresos, sean excluidos de áreas en proceso de valorización, acelerando fenómenos de gentrificación y desplazamiento de comunidades.

\subsection*{Burbujas inmobiliarias y volatilidad en el mercado}
Las malas estimaciones de precios pueden contribuir a la formación de burbujas inmobiliarias, donde los precios se inflan artificialmente debido a una sobrevaloración de las propiedades. Cuando la burbuja estalla, los precios caen bruscamente, lo que provoca una crisis de confianza en el mercado y pérdidas económicas significativas para propietarios e inversionistas. Este tipo de volatilidad afecta la estabilidad financiera de las familias y las inversiones de largo plazo.

\subsection*{Dificultades para la planificación urbana}
Una buena estimación de precios es clave para la planificación urbana y la asignación de recursos en infraestructura, servicios públicos y desarrollo sostenible. Si las predicciones de precios no reflejan adecuadamente el valor futuro de las propiedades, los gobiernos locales y desarrolladores pueden tomar decisiones incorrectas sobre dónde invertir en infraestructura y servicios. Esto puede resultar en zonas sobrepobladas sin servicios adecuados o en áreas subdesarrolladas que no reciben suficiente inversión.

\subsection*{Pérdida de confianza de inversionistas}
Los inversionistas dependen de estimaciones precisas de precios para identificar oportunidades de crecimiento en diferentes zonas. Si los modelos utilizados no logran prever con exactitud la evolución de los precios, los inversionistas podrían enfrentar pérdidas o no alcanzar la rentabilidad esperada. Esto desalienta la inversión en el sector inmobiliario y puede impactar negativamente en el desarrollo económico local.

\subsection*{Impacto en el crédito hipotecario}
Las instituciones financieras basan sus decisiones de otorgamiento de crédito hipotecario en evaluaciones precisas del valor de las propiedades. Cuando estas evaluaciones están basadas en estimaciones incorrectas, el riesgo de impagos o de tener propiedades sobrevaluadas en el portafolio de préstamos aumenta. Esto puede derivar en pérdidas para los bancos y en restricciones más severas para otorgar créditos, afectando a las familias que buscan adquirir vivienda.

La correcta predicción de precios, basada en datos estructurados y enriquecidos, no solo mejorará la transparencia y eficiencia del mercado inmobiliario en Bogotá, sino que también evitará estos problemas derivados. Por lo tanto, el desarrollo de un modelo de predicción que incorpore tanto características internas de las propiedades como datos externos (seguridad, cercanía a servicios, etc.) podría ser fundamental para la sostenibilidad y el crecimiento del mercado inmobiliario, promoviendo una mayor equidad y una mejor toma de decisiones por parte de todos los actores involucrados.

    
    % Metodologia
    \chapter{Metodolog\'ia de investigación}
    \section*{Metodología}
Durante el desarrollo del proyecto se estructuró un proceso reproducible para estimación 
de precios de vivienda que incluyó:
\begin{enumerate}
    \item Adquisición y validación de datos.
    \item Limpieza con reglas explícitas para mitigar sesgos por valores extremos y errores de captura.
    \item Entrenamiento de un modelo de línea base.
    \item Enriquecimiento geoespacial con capas oficiales y POIs.
    \item Entrenamiento y evaluación de modelos con validación cruzada.
    \item Cálculo de valores estadísticos agrupados para enriquecer resultados.
    \item Persistencia de artefactos y exposición mediante API.
\end{enumerate}


La métrica principal seleccionada fue \textbf{RMSE} (raíz del error cuadrático medio) por su interpretabilidad en la misma unidad del objetivo (pesos colombianos) y su mayor penalización de errores grandes, deseable en valuaciones. Se reportan además \textbf{MAE} como medida robusta del error absoluto y \(\mathbf{R^2}\) para el ajuste explicativo.


\begin{figure}[h]
    \centering
    \includegraphics[width=0.85\linewidth]{Images/metodologia.png}
    \caption{Proceso general de la metodología}
    \label{fig:metodologia}
\end{figure}

\subsection*{Fuentes de datos}
Se integraron fuentes públicas y abiertas. 
Los datos de inmuebles proveen variables estructurales, mientras que las capas distritales y POIs agregan contexto espacial y de accesibilidad.
\begin{itemize}
    \item \textbf{Datos de inmuebles (base)}: JSON único de agosto de 2024 publicado en GitHub (\texttt{builker-col/bogota-apartments}). Se transformó a CSV para su análisis.
    \item \textbf{Datos abiertos del distrito (PostGIS)}: Se obtuvieron capas desde la web de datos abiertos del Distrito \cite{datosabiertos_bogota} 
    (consultada en julio de 2025), estos archivos fueron descargados en formato .shp y 
    luego fueron cargados como capas en postgis mediante un script de importación, las capas
    descargados fueron las siguientes: \texttt{barrios\_bogota}, \texttt{upz\_bogota}, 
    \texttt{localidades\_bogota}, 
    \texttt{estratos\_manzana}, \texttt{avaluo\_catastral\_manzana}, 
    y POIs OSM (\texttt{gis\_osm\_pois\_free\_1}\allowbreak, \texttt{gis\_osm\_pois\_a\_free\_1}).
    \item \textbf{Puntos de interés (POIs)}:
    Los puntos de interés cercanos se obtuvieron a partir de datos de OpenStreet map, 
    descargados desde la página web de Geofabrik \cite{geofabrik}.
\end{itemize}



\subsection*{Limpieza y preprocesamiento}
Se normalizó la calidad del dataset para reducir la influencia de colas largas y errores de captura. Se imputaron valores faltantes preservando coherencia estructural entre variables comparables. El proceso se ejecutó en Python (\textit{Pandas}, \textit{NumPy}, \textit{scikit-learn}).
\begin{itemize}
    \item \textbf{Outliers}: filtro por percentil 99. \emph{Área} \(\leq 464\,m^2\), \emph{precio\_venta} \(\leq 5{,}4\times 10^9\) COP.
    \item \textbf{Precio mínimo}: \(\geq 50{,}000{,}000\) COP.
    \item \textbf{Área igual a 0}: imputación por mediana de comparables (mismo \emph{estrato}, \emph{habitaciones}, \emph{banos}, \emph{sector}); si no hay, mediana por \emph{estrato}.
    \item \textbf{Parqueaderos negativos}: reemplazo por moda dentro del mismo \emph{estrato} (sino, moda global).
    \item \textbf{Coordenadas}: imputación por mediana del \emph{sector} o global cuando estén fuera de Bogotá; bounding box final: lat \([4.4, 4.9]\), lon \([-74.3, -73.9]\). Registros fuera se eliminan tras imputación.
    \item \textbf{Estrato fuera de rango [1--6]}: imputación por modo del \emph{sector}; si no hay, modo global.
\end{itemize}

\subsection*{Modelos base}
Se evaluaron varios algoritmos con una misma estrategia de preprocesamiento para asegurar comparabilidad. La validación cruzada controló varianza y evitó sobreajuste; la métrica principal fue RMSE en escala real.
Preprocesamiento: \texttt{SimpleImputer} (media/moda), \texttt{StandardScaler}, \texttt{OneHotEncoder}. Resultados (aprox.):
\begin{itemize}
    \item \textbf{Random Forest}: RMSE \(\approx\) 245M (CV); en\ hold-out (20\%): RMSE \(\approx\) 250.6M, MAE \(\approx\) 129.9M, R\textsuperscript{2} \(\approx\) 0.915.
    \item \textbf{XGBoost / LightGBM}: RMSE \(\approx\) 245--246M (CV).
    \item \textbf{Lineales}: \(\approx\) 348M; \textbf{SVR}: \(\approx\) 913M.
\end{itemize}
Se evaluó además con \(\log(\textit{precio\_venta})\) y \(\log(\textit{area})\), mejorando métricas en escala original.

Con base en la línea base, se desarrolló un \emph{modelo aumentado} incorporando variables de contexto espacial (POIs por radio, metadatos regionales y valuación por geohash) para capturar efectos de accesibilidad y localización.

\subsection*{Uso de Geohash}
Para optimizar el cálculo de puntos cercanos, se utilizó el geohash con un nivel de precisión que 
asegurara una granularidad adecuada para agrupar sectores similares sin comprometer el rendimiento. 
Se seleccionó un nivel de precisión 7, que genera cuadrículas de aproximadamente 
152.59 metros \cite{geohash_size}.

\subsection*{Enriquecimiento geoespacial}
Se incorporó información de entorno con el fin de capturar efectos de accesibilidad y amenidades. 
El cómputo se realizó por geohash y luego cada propiedad se ubicó dentro de cada geohash para 
agregar variables como conteos de POIs OSM, metadatos regionales y valuación por geohash.
\begin{itemize}
    \item \textbf{Conteos de POIs OSM}: radios de 100, 300, 500, 1000 y 2000 metros en categorías agregadas: \emph{education}, \emph{healthcare}, \emph{retail\_access}, \emph{dining\_and\_entertainment}, \emph{accommodation}, \emph{parks\_and\_recreation}, \emph{infrastructure\_services}, \emph{cultural\_amenities}.
    \item \textbf{Metadatos regionales}: asignación de UPZ, barrio y localidad; variables \emph{upz\_calculada}, \emph{barrio\_calculado}, \emph{localidad\_calculada}.
    \item \textbf{Valuación por geohash}: promedios de \emph{catastral} y \emph{comercial} por celda (geohash), usando \texttt{avaluo\_catastral\_manzana}.
    \item \textbf{Persistencia y estadísticas}: escritura en \texttt{property\_data} y agregación en \texttt{region\_stats} (barrio, UPZ, localidad) vía \texttt{ST\_Contains}, con \(n\), medias, desviaciones y cuartiles.
\end{itemize}


\subsection*{Modelos con datos aumentados}
Se entrenaron variantes con variables enriquecidas y selección reducida para medir el aporte incremental del contexto espacial.
\begin{itemize}
    \item \textbf{v0 (XGB)}: \(\log(\textit{area})\), one-hot; variables estructurales dominan; enriquecidas con aporte marginal.
    \item \textbf{v1 (XGB reducido + barrio\_top)}: hold-out RMSE \(\approx\) 254.66M, MAE \(\approx\) 136.54M, R\textsuperscript{2} \(\approx\) 0.9139. Modelo exportado como \texttt{xgboost\_model\_2.1.pkl}.
    \item \textbf{v2 (XGB con búsqueda aleatoria)}: mejores hiperparámetros: \(n\_\textit{estimators}=500\), \(\textit{max\_depth}=9\), \(\textit{learning\_rate}=0.05\), \(\textit{subsample}=0.8\), \(\textit{colsample\_bytree}=0.8\), \(\alpha=0\), \(\lambda=1\). Hold-out: RMSE \(\approx\) 233.49M, MAE \(\approx\) 121.96M, R\textsuperscript{2} \(\approx\) 0.9276. Exportado como \texttt{xgboost\_model\_2.2.pkl}.
\end{itemize}

\subsection*{Exposición de resultados y API}
Se expuso un endpoint \texttt{GET /api/estimate} (FastAPI) que integra estadísticos del punto (POIs, región, valuación) y la predicción del modelo. El servicio agrega \texttt{get\_point\_stats}, \texttt{get\_region\_stats} y \texttt{estimate(...)} sobre la versión de modelo configurada (por defecto 2.1; se recomienda 2.2 por mejor RMSE).

\subsection*{Reproducibilidad}
Se documentó el flujo completo para su ejecución independiente. Los notebooks en \texttt{analisis/notebooks/} contienen extracción, limpieza, EDA y entrenamiento; el proyecto \texttt{indexador-py/} encapsula el enriquecimiento y la persistencia en PostGIS. Se fijaron semillas aleatorias y la configuración de validación para replicar métricas. Los modelos serializados y datasets intermedios se registraron en \texttt{analisis/data/}.


    
    % Cronograma
    \chapter{Resultados y An\'alisis}
    \section{Resultados}
\subsection{Modelo base}
Se evaluaron múltiples algoritmos con validación cruzada (KFold=5, \texttt{shuffle=True}, \texttt{random\_state=42}), usando RMSE en escala real. El mejor desempeño base lo obtuvo \textbf{Random Forest}:
\begin{itemize}
    \item \textbf{CV (aprox)}: RF \(\approx\) 245M; XGB \(\approx\) 245M; LGBM \(\approx\) 246M; lineales \(\approx\) 348M; SVR \(\approx\) 913M.
    \item \textbf{Hold-out (20\%)}: RF RMSE \(\approx\) 250.6M; MAE \(\approx\) 129.9M; R\textsuperscript{2} \(\approx\) 0.915.
\end{itemize}
El uso de transformaciones \(\log(\textit{precio\_venta})\) y \(\log(\textit{area})\) mejoró las métricas al regresar a la escala original.

\subsection{Modelos con datos aumentados}
Tras enriquecer con variables contextuales (POIs por radio, asignación de UPZ/barrio/localidad, avalúos por geohash) se entrenaron variantes XGBoost:
\begin{itemize}
    \item \textbf{v1 (reducido + \textit{barrio\_top})}: RMSE \(\approx\) 254.66M; MAE \(\approx\) 136.54M; R\textsuperscript{2} \(\approx\) 0.9139. Modelo exportado: \texttt{xgboost\_model\_2.1.pkl}.
    \item \textbf{v2 (búsqueda aleatoria)}: hiperparámetros: \(n\_\textit{estimators}=500\), \(\textit{max\_depth}=9\), \(\textit{learning\_rate}=0.05\), \(\textit{subsample}=0.8\), \(\textit{colsample\_bytree}=0.8\), \(\alpha=0\), \(\lambda=1\). Hold-out: RMSE \(\approx\) 233.49M; MAE \(\approx\) 121.96M; R\textsuperscript{2} \(\approx\) 0.9276. Modelo exportado: \texttt{xgboost\_model\_2.2.pkl}.
\end{itemize}

\subsection{Importancia de variables y hallazgos}
\begin{itemize}
    \item \textbf{Variables dominantes}: \emph{área}, \emph{latitud}, \emph{administración}, \emph{antigüedad}, \emph{longitud} aparecen consistentemente entre las más relevantes.
    \item \textbf{Contexto espacial}: las variables de POIs por radio aportan de forma complementaria; su efecto es menor que el de las estructurales pero mejora el ajuste en v2.
    \item \textbf{Distribución del error}: el percentil 80 del error absoluto se ubica en torno a \(\sim\)187M para el RF base; v2 reduce RMSE y MAE manteniendo estabilidad.
\end{itemize}

\subsection{Artefactos y reproducibilidad}
\begin{itemize}
    \item Modelos serializados en \texttt{analisis/data/models/}: \texttt{randomforest\_model\_base.pkl}, \texttt{xgboost\_model\_2.1.pkl}, \texttt{xgboost\_model\_2.2.pkl}.
    \item Dataset enriquecido: \texttt{analisis/data/aptos\_bogota\_enriched.csv}.
    \item Pipeline de estimación expuesto vía \texttt{GET /api/estimate} en el backend (ver metodología).
\end{itemize}


    
    % Presupuesto
    \chapter{Discusi\'on}
    El presente estudio tuvo como propósito evaluar el impacto del enriquecimiento geoespacial en la predicción de precios de vivienda en Bogotá, analizando si la incorporación de variables contextuales (proximidad a servicios, equipamientos urbanos y características de entorno) mejoraba la capacidad explicativa de un modelo de aprendizaje automático respecto a uno construido únicamente con variables estructurales. Los resultados mostraron que, aunque las variables espaciales no incrementaron de forma sustancial la precisión del modelo, sí aportaron información valiosa para interpretar la lógica territorial del mercado inmobiliario.

El modelo base, entrenado con variables estructurales y administrativas, presentó un desempeño destacado y estable, con un RMSE promedio de 0.151 y un $R^2$ de 0.965. Este resultado confirma que los atributos físicos del inmueble —principalmente el área, el estrato, el número de parqueaderos y la antigüedad— concentran la mayor parte de la varianza explicativa del precio. La validación cruzada de diez pliegues evidenció una baja dispersión en los errores, lo cual indica robustez y ausencia de sobreajuste. Asimismo, la transformación logarítmica aplicada al precio y al área permitió estabilizar la varianza y mejorar la linealidad de las relaciones, contribuyendo a la calidad del ajuste.

El comportamiento del modelo base refleja, en gran medida, las dinámicas reales de valoración utilizadas por agentes inmobiliarios y portales especializados. En la práctica, las inmobiliarias y asesores comerciales suelen estimar los precios de publicación a partir del \textit{precio por metro cuadrado} de propiedades comparables en la misma zona, ajustando ligeramente por características como número de baños o estado de conservación. Este procedimiento, basado en comparaciones locales, implica que las variables estructurales y socioeconómicas ya contienen implícitamente gran parte de la información espacial que el modelo podría aprender.

Desde esta perspectiva, el alto rendimiento del modelo base no solo valida su solidez técnica, sino que también revela la racionalidad del mercado: la ubicación y el tamaño explican la mayor parte del precio porque reflejan las prácticas habituales de valoración en el sector inmobiliario.

\subsection*{Aporte y limitaciones de las variables geoespaciales}

Aunque las variables geoespaciales no produjeron una mejora significativa en el RMSE, su inclusión permitió una interpretación más rica del entorno urbano. Sin embargo, el efecto predictivo limitado puede explicarse por varios factores:

\begin{itemize}
    \item \textbf{Redundancia espacial:} los inmuebles cercanos tienden a compartir características similares de accesibilidad y entorno. Al utilizar radios fijos (100 a 2000 metros) para el cálculo de puntos de interés, muchos registros presentaron conteos de POIs prácticamente idénticos, lo que redujo la variabilidad informativa.
    \item \textbf{Segmentación del mercado:} los estratos socioeconómicos en Bogotá ya actúan como proxy espacial del acceso a servicios y equipamientos, haciendo que las variables de entorno sean parcialmente redundantes.
    \item \textbf{Resolución de los datos:} la agregación por geohash de nivel 7 (aprox. 150 metros) ofrece una granularidad adecuada, pero puede no captar microdiferencias relevantes en áreas con alta densidad de servicios o variabilidad de precios.
\end{itemize}

A pesar de estas limitaciones, las variables geoespaciales aportaron valor interpretativo. La importancia asignada a la latitud y longitud dentro del modelo reducido sugiere que la localización sigue siendo un componente estructural del valor, aunque su efecto marginal sea menor una vez controladas las variables internas del inmueble.

Un hallazgo destacable del modelo reducido es la permanencia de la variable \texttt{upz}, lo cual resulta coherente con el diseño y la función de las Unidades de Planeamiento Zonal (UPZ) en la estructura urbana de Bogotá. Las UPZ constituyen divisiones territoriales definidas por el Distrito Capital que agrupan sectores con características socioeconómicas, morfológicas y de equipamiento relativamente homogéneas. 

La prevalencia de esta variable en el modelo sugiere que la \textbf{UPZ actúa como un nivel intermedio de agregación espacial} capaz de sintetizar información contextual que de otro modo requeriría múltiples variables geoespaciales. En efecto, los límites de las UPZ tienden a coincidir con zonas de uso del suelo, densidad y estrato similares, lo que las convierte en un indicador robusto del entorno urbano y de su valoración relativa dentro del mercado inmobiliario.

Este resultado refuerza la idea de que la estructura socioespacial de la ciudad se encuentra implícitamente codificada en la variable \texttt{upz}, y que su incorporación permite al modelo capturar las diferencias territoriales sin necesidad de un número excesivo de variables de proximidad o conteo de servicios. Desde un punto de vista práctico, la UPZ puede considerarse una \textbf{unidad de análisis óptima} para la modelación inmobiliaria en Bogotá, al balancear granularidad espacial, estabilidad administrativa y coherencia socioeconómica.

\subsection*{Sesgos y representatividad de los datos}

Un aspecto relevante para comprender los resultados es la naturaleza de los datos utilizados. Los portales inmobiliarios operan bajo esquemas de pago por publicación: las inmobiliarias adquieren paquetes que les permiten listar un número limitado de inmuebles por mes. En consecuencia, tienden a priorizar la publicación de propiedades de mayor valor, que generan comisiones más altas y justifican mejor la inversión publicitaria. Este comportamiento introduce un sesgo estructural en la muestra, sobrerrepresentando inmuebles de estratos medios y altos, y subrepresentando zonas periféricas o de menor valor.

Adicionalmente, en los estratos más bajos de la ciudad es frecuente que las transacciones inmobiliarias se realicen de forma directa entre comprador y vendedor, sin la intervención de agentes o empresas inmobiliarias. Esta práctica responde tanto a la informalidad del mercado como a la necesidad de evitar el pago de comisiones, lo que excluye del registro digital una proporción significativa de operaciones reales. En consecuencia, los portales inmobiliarios —cuyos principales clientes son precisamente las agencias y corredores— reflejan de manera más completa la oferta de vivienda en sectores de estrato 3 en adelante, pero no en los niveles socioeconómicos inferiores.

Este sesgo puede explicar por qué los modelos —tanto el base como el enriquecido— alcanzan un rendimiento mayor en los sectores de alta valorización y presentan errores más altos en áreas con menor presencia de datos. En otras palabras, el modelo no solo refleja patrones del mercado inmobiliario, sino también las dinámicas de visibilidad y acceso a la información dentro de los portales digitales, condicionadas por la estructura comercial del sector.

Durante el proceso de optimización y reducción de características, algunas variables que intuitivamente podrían considerarse relevantes fueron excluidas del modelo final por presentar una contribución estadísticamente marginal. Entre ellas destacan el número de habitaciones (\texttt{habitaciones}) y la presencia de ascensor (\texttt{ascensor}). 

La ausencia de la variable \texttt{habitaciones} puede interpretarse en el contexto del mercado inmobiliario local, donde el \textbf{área total} actúa como principal indicador del tamaño y, por tanto, captura indirectamente la influencia del número de habitaciones. En los portales inmobiliarios y en la práctica de tasación informal, los compradores y agentes suelen evaluar el valor de un inmueble principalmente con base en el precio por metro cuadrado, lo que reduce la sensibilidad del modelo a la cantidad de habitaciones.

De forma similar, la variable \texttt{ascensor}, aunque puede tener un impacto perceptible en el valor de los pisos altos o bajos, no mostró una importancia significativa en el modelo reducido. Esto podría deberse a que su efecto se diluye dentro del conjunto de características estructurales y de localización, o bien a que la relación entre el precio y la altura del piso no se encuentra suficientemente representada en los datos disponibles. En edificaciones sin ascensor, por ejemplo, el valor tiende a concentrarse en los pisos inferiores, mientras que en aquellas que sí cuentan con ascensor, el precio se incrementa en los pisos superiores, pero esta variabilidad no se refleja en la información disponible.

Estas observaciones sugieren que el modelo replica las prácticas reales del mercado, donde la valoración de los inmuebles se concentra en métricas simples como el área o el estrato, dejando de lado factores arquitectónicos o de confort que, aunque relevantes, no siempre están explícitamente incorporados en las bases de datos disponibles.

Otro aspecto relevante observado durante el análisis del modelo reducido es que ninguna de las variables añadidas durante el proceso de enriquecimiento geoespacial fue finalmente retenida en el conjunto de predictores seleccionados. A excepción de la variable \texttt{barrio}, el modelo reducido se compone exclusivamente de atributos estructurales y administrativos del inmueble. Este resultado sugiere que la información espacial incorporada mediante los conteos de puntos de interés (POIs) o las variables agregadas por geohash podría encontrarse \textbf{implícitamente representada} en las características originales de los inmuebles o en el contexto socioeconómico de cada barrio.

En otras palabras, variables como la accesibilidad, la proximidad a servicios o la densidad de equipamientos urbanos pueden estar ya codificadas de forma indirecta a través de variables correlacionadas, como el estrato o la localización del barrio. Este fenómeno es consistente con la estructura espacial del mercado inmobiliario en Bogotá, donde las zonas de mayor valorización tienden a concentrar tanto los servicios urbanos como los precios más altos, generando colinealidad entre variables estructurales y espaciales.

Como línea de trabajo futuro, se plantea la posibilidad de desarrollar \textbf{modelos especializados por sector o localidad}, que permitan capturar las particularidades de cada zona y reducir el error predictivo en áreas con comportamientos de mercado más heterogéneos. Esta estrategia podría mejorar la precisión global del sistema, al tiempo que permitiría identificar factores diferenciales de valorización entre submercados urbanos específicos.

\subsection*{Análisis de error y generalización}

El análisis de error mostró que el modelo mantiene un comportamiento estable a lo largo de los diferentes rangos de precios: el error relativo promedio se mantiene entre 3\% y 4\%. Las desviaciones mayores se concentran en zonas de alta valorización (Chicó, Santa Cecilia, Cerros de Suba), donde intervienen factores cualitativos no registrados, como acabados, vista panorámica o reputación del edificio. En contraste, los sectores intermedios y periféricos presentaron un error más uniforme, lo que sugiere que el modelo generaliza bien en contextos homogéneos.

Este patrón refuerza la idea de que la mayor parte de la varianza explicativa proviene de características estructurales y de mercado, mientras que las variables de entorno aportan información secundaria más útil para análisis espaciales que para predicción directa.

Los resultados demuestran que los modelos basados en aprendizaje automático pueden replicar con alta fidelidad las dinámicas de valoración observadas en el mercado inmobiliario. El enriquecimiento geoespacial, aunque no mejora de forma significativa las métricas de precisión, amplía la interpretabilidad del modelo y permite visualizar los factores contextuales que influyen en la formación de precios. 

De cara a futuras investigaciones, se sugiere:
\begin{itemize}
    \item Incrementar la granularidad espacial (por manzana o código catastral) y temporal (series trimestrales o mensuales).
    \item Incorporar variables de conectividad real, como tiempos de desplazamiento o accesibilidad multimodal.
    \item Integrar fuentes complementarias de datos económicos (tasa de valorización, dinámica de oferta y demanda).
    \item Explorar modelos híbridos o espaciales avanzados (GWR, \textit{graph neural networks}) para capturar autocorrelaciones geográficas.
    \item Desarrollar modelos específicos por sector o localidad que permitan capturar patrones propios de cada submercado urbano.
\end{itemize}

En síntesis, los resultados reflejan tanto la estructura técnica del modelo como las lógicas de mercado que determinan el precio de la vivienda en Bogotá: el área y la ubicación siguen siendo los principales determinantes del valor, mientras que el contexto geográfico aporta profundidad analítica sin alterar de manera sustancial la precisión global del sistema.
    
    % Resultados esperados
    \chapter{Conclusiones y Trabajo Futuro}
    
% Incluya los resultados esperados del desarrollo del proyecto de investigación en el marco del programa de Maestría en Inteligencia Artificial desde tres perspectivas: 

\begin{table}[htbp!]
    \centering
    \caption{Resultados o productos esperados de nuevo conocimiento.}
    \pgfplotstabletypeset[
	  col sep=colon, 
	  trim cells=true,
	  column type={|p{0.30\textwidth}|p{0.40\textwidth}|p{0.30\textwidth}}, 
	  columns/Resultado/.style = {column name={\cellcolor{AzulInstitucional}\textbf{\color{Blanco}Resultado o Producto esperado}}}, 
	  columns/Indicador/.style = {column name={\cellcolor{AzulInstitucional}\textbf{\color{Blanco}Indicador }}},
	  columns/Beneficiario/.style = {column name={\cellcolor{AzulInstitucional}\textbf{\color{Blanco}Beneficiario}}},
	  every head row = { \hline},
	  after row = {\hline},
	  string type,
	  skip rows between index={10}{20}
	]
	{Tables/DataNuevoConocimiento.csv}
    \label{tab:resEspNuecoConocimiento}
\end{table}


La anterior tabla resume los resultados esperados del actual proyecto desde tres perspectivas clave:

\begin{enumerate}
    \item Generación de nuevo conocimiento
    \item Fortalecimiento de la capacidad científica nacional
    \item Apropiación social del conocimiento
\end{enumerate}

Cada resultado esperado se acompaña de indicadores específicos para medir su éxito y los beneficiarios principales que se verán impactados positivamente por el proyecto.

% A continuaci\'on se presenta una tablas para la consolidaci\'on de estos resultados esperados a manera de ejemplo.





    
    % Anexos
    \chapter*{Anexos}
    % \section{Manual de instalación}

% \section{Manual de usuario}

% \section{Informe de ejecución de pruebas}

\section{Anexo técnico}

\subsection{Capas PostGIS y SRID}
\begin{itemize}
    \item \textbf{Capas}: \texttt{barrios\_bogota}, \texttt{upz\_bogota}, \texttt{localidades\_bogota}, \texttt{estratos\_manzana}, \texttt{avaluo\_catastral\_manzana}, \texttt{gis\_osm\_pois\_free\_1}, \texttt{gis\_osm\_pois\_a\_free\_1} \cite{datosabiertos_bogota}.
    \item \textbf{SRID}: 4326 (WGS84). Para consultas por distancia se usa proyección a 3857 cuando aplica (\texttt{ST\_Transform}).
    \item \textbf{Consultas típicas}: \texttt{ST\_DWithin}, \texttt{ST\_Contains}, \texttt{ST\_Intersects}, centroids y buffers en metros.
\end{itemize}

\subsection{Sistemas de referencia: WGS84 (EPSG:4326) y Web Mercator (EPSG:3857)}\label{annex:crs}
\begin{itemize}
    \item \textbf{WGS84 (EPSG:4326)}: sistema geodésico global usado por GPS. \emph{Coordenadas en grados} (latitud/longitud). Ventaja: interoperabilidad y exactitud posicional. Limitación: los grados no son métricos; 1$^{\circ}$ de longitud equivale a distintas distancias según la latitud.
    \item \textbf{Web Mercator (EPSG:3857)}: proyección métrica popular para mapas web. \emph{Coordenadas en metros} (pseudo-mercator). Ventaja: permite cálculos de distancia y \texttt{ST\_DWithin} en \emph{metros}. Limitación: distorsiona áreas y distancias al alejarse del ecuador (aceptable para escalas urbanas como Bogotá).
    \item \textbf{Cuándo usar cada uno}: almacenar y cruzar capas administrativas en 4326; proyectar a 3857 para consultas con radios en metros o buffers métricos (\texttt{ST\_Transform(geom, 3857)}).
    \item \textbf{Implicaciones en PostGIS}: para \texttt{ST\_DWithin} con radio en metros, asegure que ambas geometrías estén en 3857; para \texttt{ST\_Contains}/\texttt{ST\_Intersects} topológicos, 4326 es suficiente.
\end{itemize}

\subsection{Reglas de limpieza}
\begin{itemize}
    \item \textbf{Outliers (p99)}: \emph{área} \(\leq 464\,m^2\), \emph{precio\_venta} \(\leq 5{,}4\times10^9\) COP.
    \item \textbf{Precio mínimo}: \(\geq 50{,}000{,}000\) COP.
    \item \textbf{Área = 0}: mediana de comparables (\emph{estrato}, \emph{habitaciones}, \emph{banos}, \emph{sector}); si no hay, mediana por \emph{estrato}.
    \item \textbf{Parqueaderos < 0}: reemplazo por moda del mismo \emph{estrato}; si no hay, moda global.
    \item \textbf{Coordenadas}: imputación por mediana del \emph{sector} y filtro final a Bogotá: lat \([4.4,4.9]\), lon \([-74.3,-73.9]\).
    \item \textbf{Estrato fuera [1--6]}: imputación por modo del \emph{sector}; si no hay, modo global.
\end{itemize}

\subsection{Enriquecimiento geoespacial}
\begin{itemize}
    \item \textbf{Conteos OSM por radio}: 100, 300, 500, 1000, 2000 m.
    \item \textbf{Categorías}: \emph{education}, \emph{healthcare}, \emph{retail\_access}, \emph{dining\_and\_entertainment}, \emph{accommodation}, \emph{parks\_and\_recreation}, \emph{infrastructure\_services}, \emph{cultural\_amenities}.
    \item \textbf{Región calculada}: \emph{upz\_calculada}, \emph{barrio\_calculado}, \emph{localidad\_calculada} por \texttt{ST\_Contains}.
    \item \textbf{Avalúos por geohash}: promedios de \emph{catastral} y \emph{comercial} en bbox del geohash sobre \texttt{avaluo\_catastral\_manzana}.
    \item \textbf{Persistencia}: tabla \texttt{property\_data} adaptada al DF y agregados \texttt{region\_stats} (barrio/UPZ/localidad) con \(n\), medias, desviaciones y cuartiles.
\end{itemize}

\subsection{Modelos, métricas e hiperparámetros}
\begin{itemize}
    \item \textbf{Validación}: KFold=5 (\texttt{shuffle=True}, \texttt{random\_state=42}); métrica principal RMSE (escala real).
    \item \textbf{Base}: RF \(\approx\) 245M (CV); hold-out (20\%): RMSE \(\approx\) 250.6M, MAE \(\approx\) 129.9M, R\textsuperscript{2} \(\approx\) 0.915.
    \item \textbf{Aumentados}: v1 (XGB reducido + \emph{barrio\_top}) RMSE \(\approx\) 254.66M; v2 (XGB tuning) RMSE \(\approx\) 233.49M.
    \item \textbf{Hiperparámetros v2}: \(n\_\textit{estimators}=500\), \(\textit{max\_depth}=9\), \(\textit{learning\_rate}=0.05\), \(\textit{subsample}=0.8\), \(\textit{colsample\_bytree}=0.8\), \(\alpha=0\), \(\lambda=1\).
\end{itemize}

    
    % Indices
    % \chapter*{\'Indices}
    % \input{BackMatter/B10-Index}
    
	% Incluye las referencias
    \bibliographystyle{apalike}
    \bibliography{Bibliography/referencias}

\end{document}
